%%%%%%%%%%%%%%%%%%%%%%%%%%%%%%%%%%%%%%%%%%%%%%%%%%%%%%%%%%%%%%%%%%%%%%%%%%%%%%
% ASU Dissertation Template
%%%%%%%%%%%%%%%%%%%%%%%%%%%%%%%%%%%%%%%%%%%%%%%%%%%%%%%%%%%%%%%%%%%%%%%%%%%%%%
% Copyright 2018 Robert W. Kutter (robert@kutterconsulting.com)
%
%   See also: http://kutterconsulting.com
%
% For guidance on using this file, see the README.
%
%%%%%%%%%%%%%%%%%%%%%%%%%%%%%%%%%%%%%%%%%%%%%%%%%%%%%%%%%%%%%%%%%%%%%%%%%%%%%%
% Preamble
%%%%%%%%%%%%%%%%%%%%%%%%%%%%%%%%%%%%%%%%%%%%%%%%%%%%%%%%%%%%%%%%%%%%%%%%%%%%%%
\newcommand*{\pointsize}{12pt}          %<Set the font size; make sure the size is correct
                                        %   for the font you will use
\documentclass[letterpaper,             % Use US letter-size paper
               oneside,                 % No verso and recto differences
               \pointsize]              % Uses the font size defined above
               {memoir}
\renewcommand{\cleardoublepage}%        % \cleardoublepage will create entirely blank
  {\clearpage}%                         %   pages depending on settings (e.g., usually
                                        %   before start of \mainmatter); redefine it here
                                        %   so that no entirely blank pages are created
                                        %   automatically
%%%%%%%%%%%%%%%%%%%%%%%%%%%%%%%%%%%%%%%
% (Some) Packages
%%%%%%%%%%%%%%%%%%%%%%%%%%%%%%%%%%%%%%%
\usepackage{graphicx}                   % For importing image files
\usepackage{etoolbox}                   % For advanced commands throughout preamble
\usepackage{microtype}                  %~Improves kerning and protrusion (optional);
                                        %   See here for an introduction:
                                        %   http://www.khirevich.com/latex/microtype/

\providetoggle{usemicrotype}            % TRUE = microtype is being used
\makeatletter                           %   (Used to turn of microtype protrusion in the
\@ifpackageloaded{microtype}%           %   table of contents.)
  {\settoggle{usemicrotype}{true}}%
  {\settoggle{usemicrotype}{false}}
\makeatother
\usepackage{changepage}                 % For changing page layout (e.g., margins) in the
                                        %   middle of the document
\usepackage{calc}					              % Calculate text widths; used in page layout
                                        %   changes

%%%%%%%%%%%%%%%%%%%%%%%%%%%%%%%%%%%%%%%
% Custom Stuff
%%%%%%%%%%%%%%%%%%%%%%%%%%%%%%%%%%%%%%%
\usepackage{float}
\usepackage{xcolor}
\usepackage{minted}
\usepackage{mdframed}

\graphicspath{ {./images/} }

\newcommand*{\image}[3]{
  \begin{figure}[H]
    \begin{center}
      \includegraphics[width=0.7\maxwidth{\textwidth}]{#1}
      \caption{#2}
    \end{center}
  \end{figure}}

\newcommand*{\summaryi}[1]{%
  \textcolor{gray}{\{}\textcolor{blue}{#1}\textcolor{gray}{\}}}

\newcommand*{\summaryii}[3]{%
  \textcolor{gray}{\{}\textcolor{blue}{#1}\textcolor{gray}{, }\textcolor{blue}{#2}\textcolor{gray}{\}}}

\newcommand*{\summaryiii}[3]{%
  \textcolor{gray}{\{}\textcolor{blue}{#1}\textcolor{gray}{, }\textcolor{blue}{#2}\textcolor{gray}{, }\textcolor{blue}{#3}\textcolor{gray}{\}}}

\newenvironment{data}{
  \begin{center}
    \begin{mdframed}[backgroundcolor=lightgray]
      \begin{tiny}
        \ttfamily
        \obeyspaces
        \obeylines}{
      \end{tiny}
    \end{mdframed}
  \end{center}}

\newminted{python}{frame=lines,framesep=2mm,baselinestretch=1.2,bgcolor=lightgray,fontsize=\footnotesize,escapeinside=~~,linenos}

%%%%%%%%%%%%%%%%%%%%%%%%%%%%%%%%%%%%%%%
% Title page, input
%%%%%%%%%%%%%%%%%%%%%%%%%%%%%%%%%%%%%%%
\listadd{\titlelines}%
  {IRE}
\listadd{\titlelines}%
  {A Framework For Inductive Reverse Engineering}
\newcommand*\Author{Connor Nelson}
\newcommand*{\documentname}%
  {Thesis}
\newcommand*{\degreename}
  {Master of Science}
\newcommand*\defdate{April 2019}
\listadd{\committeechair}{Adam Doup\'e}
\newcommand*{\chairlabel}{Chair}
\listadd{\committeemember}{Yan Shoshitaishvili}
\listadd{\committeemember}{Ruoyu Wang}
\newcommand*{\gradmonth}{May}
\newcommand*{\gradyear}{2019}

\listadd{\keywords}{inductive reverse engineering}
\listadd{\keywords}{blackbox}
\listadd{\keywords}{program synthesis}
\listadd{\keywords}{program analysis}
\newcommand*{\graddate}{\gradmonth%
  \space\gradyear}

%%%%%%%%%%%%%%%%%%%%%%%%%%%%%%%%%%%%%%%
% Page layout
%%%%%%%%%%%%%%%%%%%%%%%%%%%%%%%%%%%%%%%
\settrimmedsize{\stockheight}%          % Specifies \paperheight and \paperwidth
  {\stockwidth}{*}
\settrims{0pt}{0pt}                     % Set location of page in relation to the stock.
                                        % Paper and stock size are equivalent,
                                        % so both \trimtop and \trimedge are set to 0pt
\newlength{\forfootskip}
\setlength{\forfootskip}%
  {3\baselineskip}
\newlength{\textblockheight}            % Calculate height of text block to leave room
\setlength{\textblockheight}{9.0in}     %   for footers, keeping page numbers outside
\addtolength{\textblockheight}%         %   the 1in vertical margins
  {-\forfootskip}
\settypeblocksize{\textblockheight}%    % Calculated by 1.0in vertical margins and
  {*}{*}                                %   letting margins set the width of the typeblock
\setulmargins{1.0in}{*}{*}              % Set upper margin (\uppermargin, not \topmargin);
                                        %   calculate the bottom margin
\setlrmarginsandblock{1.25in}{1.25in}{*}%~Set margins and calculate width of typeblock
\setheaderspaces{*}{0.5\baselineskip}{*}% Arguments: '\headdrop', '\headsep', and/or ratio
                                        %   Note: This is only used in the list of
                                        %   contents sections
\setheadfoot{\baselineskip}%            % Set '\headheight' and '\footskip'
  {\forfootskip}
\checkandfixthelayout                   % Required by memoir package after setting layout
\settypeoutlayoutunit{in}               % Write layout dimensions to log file in inches

%%%%%%%%%%%%%%%%%%%%%%%%%%%%%%%%%%%%%%%
% Fonts
%%%%%%%%%%%%%%%%%%%%%%%%%%%%%%%%%%%%%%%
\usepackage[T1]{fontenc}                % Standard option to handle, e.g., accented
                                        %   characters like 'ö' better
\usepackage{amssymb,mathtools}          % For AMS-LaTeX, see here for more:
                                        %     http://www.ams.org/publications/authors/tex/amslatex
                                        % ('mathtools' loads and extends 'amsmath')
\usepackage{ifxetex,ifluatex}           % Can check if XeTeX or LuaTeX was used to typeset
\usepackage{fixltx2e}                   % Provides \textsubscript
\IfFileExists{upquote.sty}%             % Use upquote if available, for
  {\usepackage{upquote}}{}              %   straight quotes in verbatim environments

% Load fonts depending on the
%   typesetting engine
\ifnum 0\ifxetex 1\fi\ifluatex 1\fi=0   % If pdftex
  \usepackage[utf8]{inputenc}           %   'utf8' should match the encoding of this file
                                        %
                                        %~Set up your font in pdftex here
                                        %
\else 									                % If xetex or luatex
  \ifxetex                              % If xetex
    \usepackage{mathspec}               % Matches non-math open-type font to math
                                        %   open-type font (use 'mathspec' if you want to
                                        %   write math in unicode)
    \usepackage{xunicode}               % Convert LaTeX character macros to unicode
  \else                                 % If luatex\usepackage{fontspec}
    \usepackage{fontspec}               % Use fontspec for (open type) font selection
  \fi
  \defaultfontfeatures{Mapping=tex-text,% Font spec setting
    Scale=MatchLowercase}
  \newcommand*{\euro}{€}
  \setmainfont{Garamond}
  \setmonofont{Source Code Pro}
\fi

%%%%%%%%%%%%%%%%%%%%%%%%%%%%%%%%%%%%%%%
% Line spacing
%%%%%%%%%%%%%%%%%%%%%%%%%%%%%%%%%%%%%%%
\DoubleSpacing                          % True double spacing
\BeforeBeginEnvironment{quote}          % Memoir leaves most special material
  {\SingleSpacing}                      %   single spaced, but makes block quotes
\AfterEndEnvironment{quote}%            %   double-spaced; fix to follow ASU style guide
  {\vspace{-\baselineskip} %
  \DoubleSpacing}
\BeforeBeginEnvironment{quotation}%
  {\SingleSpacing}
\AfterEndEnvironment{quotation}%
  {\vspace{-\baselineskip} %
  \DoubleSpacing}

\setlength{\footnotesep}{\baselineskip} % Double space *between* footnotes
\renewcommand*{\footnoterule}{%         % Redefine footnoterule so that initial footnote
  \kern-3pt%                            %   still appears right under the rule (changing
  \hrule width 0.4\columnwidth          %   \footnotesep also changes the space between the
  \kern 2.6pt                           %   rule and the first footnote
  \vspace{-0.5\baselineskip}            % (Here is the vertical space adjustment)
  }

\usepackage{enumitem}                   % Control spacing in enumerate environment
\setlist{noitemsep}                     % Remove extra vertical spacing between items in lists
                                        % \setlist{nosep} to leave no space around whole list

%%%%%%%%%%%%%%%%%%%%%%%%%%%%%%%%%%%%%%%
% Page numbering
%%%%%%%%%%%%%%%%%%%%%%%%%%%%%%%%%%%%%%%
\makepagestyle{ASU}
  \makeevenfoot{ASU}{}{\thepage}{}
  \makeoddfoot{ASU}{}{\thepage}{}

%%%%%%%%%%%%%%%%%%%%%%%%%%%%%%%%%%%%%%%
% Title page, formatting
%%%%%%%%%%%%%%%%%%%%%%%%%%%%%%%%%%%%%%%
\newlength{\savedfootskip}
\setlength{\savedfootskip}{\footskip}
\newcommand{\titlepagesetup}{%          % Page layout for title page
  \changepage%                          % Adjustment to page dimensions:
    {\savedfootskip}%                   %   text height
    {}%                                 %   text width
    {}%                                 %   even-side margin
    {}%                                 %   odd-side margin
    {}%                                 %   column sep.
    {}%                                 %   topmargin
    {}%                                 %   headheight
    {}%                                 %   headsep
    {-\savedfootskip}%                  %   footskip
}

\newcommand{\closetitlepagesetup}{%     % Undo set up for title page
  \changepage{-\savedfootskip}{}{}{}{}%
    {}{}{}{\savedfootskip}%
}

\makeatletter                           % Do not modify this section; Enter info above
\newcommand*{\titlepageASU}{
  \titlepagesetup
  \clearpage
  \begin{center}
  \SingleSpacing
  \thispagestyle{empty}
    \renewcommand*{\do}[1]{##1 \\[\baselineskip]}
    \dolistloop{\titlelines}
    by \\[\baselineskip]
    \Author \\[4\baselineskip]
    A \documentname~Presented in Partial Fulfillment \\
    of the Requirements for the Degree \\
    \degreename \\
    \vfill                              % Vertically center the portion below
    Approved \defdate~by the \\
    Graduate Supervisory Committee: \\[\baselineskip]
    \renewcommand*{\do}[1]{##1, \chairlabel \\}
    \dolistloop{\committeechair}
    \renewcommand*{\do}[1]{##1 \\}
    \dolistloop{\committeemember}
    \vfill                              % Vertically center the portion above
    ARIZONA STATE UNIVERSITY \\[\baselineskip]
    \graddate
  \end{center}
  \clearpage
  \closetitlepagesetup
}
\makeatother

%%%%%%%%%%%%%%%%%%%%%%%%%%%%%%%%%%%%%%%
% Heading styles
%%%%%%%%%%%%%%%%%%%%%%%%%%%%%%%%%%%%%%%
% Note: memoir also has \book and \part commands; do not use these
\makechapterstyle{ASU}{%                % Define chapter heading style
  \renewcommand*{\chapterheadstart}{}   % Chapter title flush with top margin
  \renewcommand*{\chapnamefont}%        % Set font for 'Chapter' or 'Appendix'
    {\normalfont}
  \renewcommand*{\chapnumfont}%         % Set font for number in chapter headings
    {\normalfont}
  \renewcommand*{\afterchapternum}%     % Insert a double line break after
    {\\[\baselineskip]}                 %   chapter number
  \renewcommand*{\chaptitlefont}%       % Set font for chapter title name
    {\normalfont}
  \setlength{\afterchapskip}{0pt}       % Set vertical space between chapter title and
                                        %   first paragraph; equivalent to one line break
                                        %   (vertical space = \afterchapskip + \baselineskip)
                                        % Note: This \afterchapskip value is only used in
                                        %   front matter
  \renewcommand*{\printchapternum}{%    % Center justify chapter number
    \centering \chapnumfont %
    \thechapter}
  \renewcommand*{\printchaptertitle}[1]%% Center justify
    {\expandafter\centering %           %   \MakeUppercase has issues; see here for some
    \expandafter\chaptitlefont %        %   details: https://tex.stackexchange.com/questions/35680/uppercase-in-newcommand
    \expandafter\MakeUppercase %        %   Accented characters and some fonts may not
    \expandafter{##1}}                  %   uppercase correctly; if that happens, just
                                        %   type the chapter title in uppercase
}

\setsecnumdepth{all}                    %~Enter the levels that you want to have numbered
                                        %   (Default is to number all [5 levels deep].)

\newcommand{\divisionbeforeskip}%       % Create default formatting for headings
  {\baselineskip}
\newcommand{\divisionindent}%
  {0.5em}
\newcommand{\divisionfont}{\normalfont} % Font must be \normalfont
\newcommand{\divisionafterskip}%
  {\baselineskip}

\setbeforesecskip{\divisionbeforeskip}  % Apply default formatting to all heading levels
\setsecindent{\divisionindent}          % Note: If you change \setsecnumdepth above, you
\setsecheadstyle{\divisionfont}         %   will need to set the indent for all lower
\setaftersecskip{\divisionafterskip}    %   levels to '0pt'; otherwise, they will be
                                        %   preceded by unnecessary space
\setbeforesubsecskip{\divisionbeforeskip}
\setsubsecindent{\divisionindent}
\setsubsecheadstyle{\divisionfont}
\setaftersubsecskip{\divisionafterskip}

\setbeforesubsubsecskip{\divisionbeforeskip}
\setsubsubsecindent{\divisionindent}
\setsubsubsecheadstyle{\divisionfont}
\setaftersubsubsecskip{\divisionafterskip}

\setbeforeparaskip{\divisionbeforeskip}
\setparaindent{\divisionindent}
\setparaheadstyle{\divisionfont}
\setafterparaskip{\divisionafterskip}

\setbeforesubparaskip{\divisionbeforeskip}
\setsubparaindent{\divisionindent}
\setsubparaheadstyle{\divisionfont}
\setaftersubparaskip{\divisionafterskip}

%%%%%%%%%%%%%%%%%%%%%%%%%%%%%%%%%%%%%%%
% Paragraph formatting
%%%%%%%%%%%%%%%%%%%%%%%%%%%%%%%%%%%%%%%
%\sloppybottom                          % Reduce the chances of widows
\raggedbottom                           % Loosens vertical spacing requirements, so
                                        %   \sloppybottom doesn't make pages look bad;
                                        %   it also prevents large gaps in the middle of
                                        %   pages and pushes them to the bottom of pages
\indentafterchapter                     % Overrides the default which is not to indent
                                        %   the first paragraph in a chapter, but it
                                        %   looks odd in some places to not indent
                                        %   paragraphs

%%% List Titles %%%
\renewcommand{\contentsname}%           % Set heading for each list
  {Table of Contents}%                  %   Formatted as chapter headings by default, so
\renewcommand{\listtablename}%          %   no additional heading formatting is needed
  {List of Tables}
\renewcommand{\listfigurename}%
  {List of Figures}

%%% Depth %%%
\settocdepth{subparagraph}              % Include 5 levels deep (all levels) in TOC

%%% Fonts %%%
\makeatletter%
\patchcmd{\l@part}%                     % Patch the command that writes part-level entries
    {\cftpartfont {#1}}%                %   to the table of contents, so they are in
    {\normalfont \texorpdfstring{%      %   'normalfont' and uppercase
      \uppercase{#1}}{{#1}} }%
    {\typeout{Success: Patch %
      'l@part' to uppercase %
      part-level headings in the %
      table of contents.}}%
    {\typeout{Fail: Patch %
      'l@part' to uppercase %
      part-level headings in the %
      table of contents.}}%
\makeatother%

\makeatletter%
\patchcmd{\l@chapapp}%                  % Patch the command that writes chapter-level
    {\cftchapterfont {#1}}%             %   entries to the table of contents, so they are
    {\normalfont \texorpdfstring{%      %   in 'normalfont' and uppercase
      \uppercase{#1}}{{#1}} }%
    {\typeout{Success: Patch %
      'l@chapapp' to uppercase %
      part-level headings in the %
      table of contents.}}%
    {\typeout{Fail: Patch %
      'l@chapapp' to uppercase %
      part-level headings in the %
      table of contents.}}%
\makeatother%

% If not using 'hyperref', use the following commands to adjust 'part' and 'chapter'
%   level headings in the TOC
%\renewcommand*{\cftpartfont}%          % Uppercase 'part' and 'chapter' headings
%  {\normalfont\MakeTextUppercase}      % Note: Sending \MakeTextUppercase to the TOC
%\renewcommand*{\cftchapterfont}%       %   conflicts with hyperref and breaks it!
%  {\normalfont\MakeTextUppercase}%

\usepackage{titlecaps}                  % Set up headline style for captions in the
                                        %   lists of tables and figures
                                        % Note: ASU style guide does not provide
                                        %   comprehensive guidelines for headlines, so
                                        %   Chicago style for headline style is used
                                        % Note: Last word in title is not explicitly
                                        %   capitalized; in general, these settings are
                                        %   broadly correct, but captions should be
                                        %   reviewed to ensure they are being capitalized
                                        %   properly
\Resetlcwords
\Addlcwords{a an the}                   % Leave articles lowercase
\Addlcwords{and but for or nor}         % Leave conjunctions lowercase
\Addlcwords{aboard about above across % % Leave all prepositions lowercase
  after against along amid among anti % %   (This is a [non-exhaustive] list of common
  around as at before behind below %    %   one-word prepositions)
  beneath beside besides between %
  beyond but by concerning considering %
  despite down during except excepting %
  excluding following for from in %
  inside into like minus near of off %
  on onto opposite outside over past %
  per plus regarding round save since %
  than through to toward towards under %
  underneath unlike until up upon %
  versus vs via with within without}
\Addlcwords{ according\space{to} %      % Leave two-word conjunctions lowercase
  ahead\space{of} apart\space{from} %   %   (This is a [non-exhaustive] list of common
  as\space{for} as\space{of} %          %   two-word prepositions.)
  as\space{per} as\space{regards} %
  aside\space{from} astern\space{of} %
  back\space{to} because\space{of} %
  close\space{to} due\space{to} %
  except\space{for} far\space{from} %
  in\space{to} inside\space{of} %
  instead\space{of} left\space{of} %
  near\space{to} next\space{to} %
  on\space{to} opposite\space{of} %
  opposite\space{to} out\space{from} %
  out\space{of} outside\space{of} %
  owing\space{to} prior\space{to} %
  pursuant\space{to} rather\space{than} %
  regardless\space{of} right\space{of} %
  subsequent\space{to} such\space{as} %
  thanks\space{to} that\space{of} %
  up\space{to}}

\renewcommand{\cfttableaftersnumb}%     % Put table captions in List of Tables in title
  {\titlecap}%                          %   case
\renewcommand{\cftfigureaftersnumb}%    % Put table captions in List of Figures in title
  {\titlecap}%                          %   case

\newcommand{\macrocapwrap}[1]{%         % Use this macro to place other macros inside
  {\bgroup\bgroup{{#1}}\egroup\egroup}% %   captions, e.g., '\macrocapwrap{\ref{figure1}}'
}%                                      % Note: Necessary due to the 'titlecaps' package
                                        %   which modifies contents of captions

\renewcommand*{\cftpartpagefont}%       % Use normal font for all page numbers
  {\normalfont}
\renewcommand*{\cftchapterpagefont}%
  {\normalfont}
\renewcommand*{\cftsectionpagefont}%
  {\normalfont}
\renewcommand*{\cftsubsectionpagefont}%
  {\normalfont}
\renewcommand*{\cftsubsubsectionpagefont}%
  {\normalfont}
\renewcommand*{\cftsubsubsectionpagefont}%
  {\normalfont}
\renewcommand*{\cftparagraphpagefont}%
  {\normalfont}
\renewcommand*{\cftsubparagraphpagefont}%
  {\normalfont}
\renewcommand*{\cftfigurepagefont}%
  {\normalfont}
\renewcommand*{\cfttablepagefont}%
  {\normalfont}

\cftpagenumbersoff{part}                % Turn off page numbers for 'part's, which are
                                        %   actually serving as headings within the TOC

%%% Vertical Space %%%
\setlength{\cftbeforepartskip}{0pt}     % Remove all additional vertical spacing so TOC
\setlength{\cftbeforechapterskip}{0pt}  %   is double spaced uniformly
\setlength{\cftbeforesectionskip}{0pt}
\setlength{\cftbeforesubsectionskip}{0pt}
\setlength{\cftbeforesubsubsectionskip}{0pt}
\setlength{\cftbeforeparagraphskip}{0pt}
\setlength{\cftbeforesubparagraphskip}{0pt}
\setlength{\cftbeforefigureskip}{0pt}
\setlength{\cftbeforetableskip}{0pt}

\renewcommand{\insertchapterspace}{%    % By default, extra vertical space (10pt) is
  \addtocontents{lof}%                  %   inserted between tables and figures from
    {\protect\addvspace{0pt}}%          %   different chapters; remove this extra space.
  \addtocontents{lot}%
    {\protect\addvspace{0pt}}%
}

%%% Horizontal Space %%%
\newlength{\levelindentincrement}       % Set indent to increase by the same amount for
\setlength{\levelindentincrement}{2em}  %   each level in the TOC; don't adjust figure
\newlength{\levelindent}                %   or table indents
\setlength{\levelindent}%
  {\levelindentincrement}
\setlength{\cftchapterindent}%
  {\levelindent}
\addtolength{\levelindent}%
  {\levelindentincrement}
\setlength{\cftsectionindent}%
  {\levelindent}
\addtolength{\levelindent}%
  {\levelindentincrement}
\setlength{\cftsubsectionindent}%
  {\levelindent}
\addtolength{\levelindent}%
  {\levelindentincrement}
\setlength{\cftsubsubsectionindent}%
  {\levelindent}
\addtolength{\levelindent}%
  {\levelindentincrement}
\setlength{\cftparagraphindent}%
  {\levelindent}
\addtolength{\levelindent}%
  {\levelindentincrement}
\setlength{\cftsubparagraphindent}%
  {\levelindent}
\addtolength{\levelindent}%
  {\levelindentincrement}

\setlength{\cftchapternumwidth}%        % Decrease space between number and heading for
  {0.85\cftchapternumwidth}             %   all heading levels
\setlength{\cftsectionnumwidth}%
  {0.85\cftsectionnumwidth}
\setlength{\cftsubsectionnumwidth}%
  {0.85\cftsubsectionnumwidth}
\setlength{\cftsubsubsectionnumwidth}%
  {0.85\cftsubsubsectionnumwidth}
\setlength{\cftparagraphnumwidth}%
  {0.85\cftparagraphnumwidth}
\setlength{\cftsubparagraphnumwidth}%
  {0.85\cftsubparagraphnumwidth}
\setlength{\cftfigurenumwidth}%         % Figure has the same 'level' as 'chapter' in the
  {\cftchapternumwidth}                 %   figure list, so make the number spacing the
                                        %   same as for chapters
\setlength{\cfttablenumwidth}%          % Table has the same 'level' as 'chapter' in the
  {\cftchapternumwidth}                 %   table list, so make the number spacing the
                                        %   same as for chapters

%%% Leaders/dots %%%
\renewcommand*{\cftdotsep}{1.7}         % Set distance between dots for all heading levels
\renewcommand*{\cftchapterleader}%      % Turn on dots for 'chapter' level
  {\normalfont\cftdotfill{\cftdotsep}}
\makeatletter                           % Bring leader dots over to page number (no gap)
  \renewcommand{\@pnumwidth}{1.55em}    %~Manually adjust
  \renewcommand{\@tocrmarg}{2.55em}
\makeatother

\renewcommand{\cfttableaftersnum}{.}    % Period after number in LOT
\renewcommand{\cftfigureaftersnum}{.}   % Period after number in LOF

%%% Printing List Titles and Headers in Content Lists
% Table of Contents (TOC)
\copypagestyle{ASUtoc}{ASU}%            % Page style for regular page in TOC
  \makeevenhead{ASUtoc}%
    {\leftmark}{}{Page}
  \makeoddhead{ASUtoc}%
    {\leftmark}{}{Page}

\copypagestyle{ASUtocFirst}{ASU}%       % Custom page headers for first page of TOC
  \makeevenhead{ASUtocFirst}%           %    (print out the title)
    {}%
    {\printchaptertitle{\contentsname}}%
    {}
  \makeoddhead{ASUtocFirst}%
    {}%
    {\printchaptertitle{\contentsname}}%
    {}

\renewcommand{\tocheadstart}{}%         % Usually content list titles are printed like
                                        %   chapter headings; empty that formatting

\renewcommand{\printtoctitle}[1]{}%     % Don't print TOC title using default method;
                                        %   it will be output in the header

\renewcommand{\aftertoctitle}{%         % On the first page of the TOC, print out the
  \thispagestyle{ASUtocFirst}%          %   TOC title using a custom page style and print
  \hfill Page\par%                      %   the heading for the page below in the regular
  }%                                    %   textbox
                                        % Note: Need '\par' before lists; see here: https://tex.stackexchange.com/questions/49882/yet-another-perhaps-a-missing-item-error

% List of Tables (LOT)
\copypagestyle{ASUlot}{ASU}%            % Page style for regular page in list of tables
  \makeevenhead{ASUlot}{Table}{}{Page}
  \makeoddhead{ASUlot}{Table}{}{Page}

\copypagestyle{ASUlotFirst}{ASU}%       % Custom page headers for first page of list of
  \makeevenhead{ASUlotFirst}%           %   tables (print out the title)
    {}%
    {\printchaptertitle{\listtablename}}%
    {}
  \makeoddhead{ASUlotFirst}%
    {}%
    {\printchaptertitle{\listtablename}}%
    {}

\renewcommand{\lotheadstart}{}%         % Usually content list titles are printed like
                                        %   chapter headings; empty that formatting;

\renewcommand{\printlottitle}[1]{}%     % Don't print LOT title using default method;
                                        %   it will be output in the header

\renewcommand{\afterlottitle}{%         % On the first page of the list of tables, print
  \thispagestyle{ASUlotFirst}%          %   out the title using a custom page style and
  Table\hfill Page\par}%                %   print heading below in regular textbox

% List of Figures (LOF)
\copypagestyle{ASUlof}{ASU}
  \makeevenhead{ASUlof}{Figure}{}{Page}
  \makeoddhead{ASUlof}{Figure}{}{Page}

\copypagestyle{ASUlofFirst}{ASU}%       % Custom page headers for first page of list of
  \makeevenhead{ASUlofFirst}%           %   figures (print out the title)
    {}%
    {\printchaptertitle{\listfigurename}}%
    {}
  \makeoddhead{ASUlofFirst}%
    {}%
    {\printchaptertitle{\listfigurename}}%
    {}

\renewcommand{\lofheadstart}{}%         % Usually content list titles are printed like
                                        %   chapter headings; empty that formatting

\renewcommand{\printloftitle}[1]{}%     % Don't print LOF title using default method;
                                        %   it will be output in the header

\renewcommand{\afterloftitle}{%         % On the first page of the list of figures, print
  \thispagestyle{ASUlofFirst}%          %   out the title using a custom page style and
  Figure\hfill Page\par}                %   print heading below in regular textbox

%%% Page layout (dimensions) for Contents Lists
\newlength{\verticalpush}               % Set up to change page dimensions for the table
                                        %   of contents
                                        % Push everything down so all the content is still
                                        %   1in from the top of the page, including the
                                        %   header, so the header is available for titles
                                        %   on the first page of contents lists and then
                                        %   the headings on subsequent pages
\setlength{\verticalpush}%              % Calculate difference between \headdrop and the
  {1.0in - \headdrop}                   %   total upper margin (1in), so you can push
                                        %   the top of the header down into the textbox

\newcommand{\contentslistsetup}{%       % Set up for contents lists
  \changepage%                          % Adjustment to page dimensions:
    {-\baselineskip}%                   %   text height
    {}%                                 %   text width
    {}%                                 %   even-side margin
    {}%                                 %   odd-side margin
    {}%                                 %   column sep.
    {\verticalpush}%                    %   topmargin
    {}%                                 %   headheight
    {}%                                 %   headsep
    {-\verticalpush+\baselineskip}%     %   footskip
}

\newcommand{\closecontentslistsetup}{%  % Undo set up for contents lists
  \changepage{\baselineskip}{}{}{}{}%
    {-\verticalpush}{}{}{\verticalpush-\baselineskip}%
}

% Content lists can also be output directly. If the following command were used, all the
%   headings would have to be output manually (i.e., can't rely on any memoir macros for
%   formatting or setting in contents lists headings and lists). It would be best to
%   create a custom macro, such as '\customtoc', to output headings and content lists
%   following the style guide.
%
% \makeatletter
%   \@starttoc{toc}
% \makeatother

% These pages partly explain why it's difficult to use 'afterpage' to change page layout
%   settings (essentially, it's because everything inside \afterpage has a local scope).
%   If it were possible to use 'afterpage' in that way, the content lists would  be
%   easier to format. A new page layout could be called after the first page of each
%   content  list. Instead, use page marks to get the layout required by the style guide.
% https://tex.stackexchange.com/questions/97126/attempts-to-manually-change-linewidth-ignored-by-latex
% https://tex.stackexchange.com/questions/85729/page-styles-only-work-for-thispagestyle-under-afterpage

%%%%%%%%%%%%%%%%%%%%%%%%%%%%%%%%%%%%%%%
% Footnotes and Endnotes
%%%%%%%%%%%%%%%%%%%%%%%%%%%%%%%%%%%%%%%
\usepackage{chngcntr}                   % Modify counters (e.g., for figures, footnotes)
\counterwithout*{footnote}{chapter}     % Make footnote numbering continuous throughout

\providetoggle{useendnotes}
\settoggle{useendnotes}{false}          %<Set to 'true' if you want to use endnotes
\iftoggle{useendnotes}{%                % Use the command \pagenote to create endnotes
                                        %   in the running text. They will be collected
                                        %   and printed in a 'Notes' section at the end
                                        %   of the document

  \makepagenote                         % Required in preamble if using endnotes
  \continuousnotenums                   % Numbering does *not* reset after each chapter
  \renewcommand*{\pagenotesubhead}[3]{} % No subheads inside note list (default is to
                                        %   divide them by chapter)
  \renewcommand*{\notenuminnotes}[1]%   % Remove extra space between note number and note
    {\normalfont #1.}                   %   text
  \renewcommand{\postnoteinnotes}%      % Double space *between* notes
    {\par\vspace{\baselineskip}}
}{}                                     % Do nothing here if not using endnotes

%%%%%%%%%%%%%%%%%%%%%%%%%%%%%%%%%%%%%%%
% Bibliography
%%%%%%%%%%%%%%%%%%%%%%%%%%%%%%%%%%%%%%%
\newcommand{\bibfilename}{library}%<Enter the name of the *.bib file containing the
                                        %   reference information for sources cited in
                                        %   the text. God help you if you're doing
                                        %   citations manually.
\newcommand{\bibheading}{References}    %<Enter the heading for the references section:
                                        %   'References', 'Works Cited', or 'Bibliography'

\providetoggle{usebiblatex}             % True = a biblatex package is being used;
                                        %   False = 'natbib' is being used
\settoggle{usebiblatex}{true}           %~Set to 'false' to use 'natbib' intead of
                                        %   biblatex; I strongly recommend using biblatex
                                        %   because natbib is rather old and will break
                                        %   for innocuous things like underscores in URLs
\iftoggle{usebiblatex}{%                % Settings for citation package
%                                       % Settings for 'biblatex' or a version of
%                                       %   'biblatex'
  \usepackage[style=ieee,
              backend=biber]%           % Recommend to use 'biber' instead of 'bibtex'
              {biblatex}                %~ Possibilities include: 'biblatex',
                                        %   'biblatex-apa', and 'biblatex-mla'
  \bibliography{\bibfilename}
  \setlength{\bibitemsep}%              % Set vertical distance between
    {0.5\baselineskip}%                 %   bibliography entries
  \setcounter{biburlnumpenalty}{9000}   % Break URLs in bibliography across lines
  \setcounter{biburlucpenalty}{9000}
  \setcounter{biburllcpenalty}{9000}

  \usepackage[style=american,%          % Settings for quotation marks; load after
    english=american]{csquotes}%        %   'inputenc'; only use with biblatex; throws
  \MakeOuterQuote{"}%                   %   error when used with natbib
}{%                                     % Settings for 'natbib'
  \usepackage{natbib}%
  \newcommand{\natbibstyle}{asudis}%    %~Enter the name of the *.bst file to use to
                                        %   format citations with natbib. Default is
                                        %   'asudis'. I do not know where 'asudis' came
                                        %   from, but apparently it formats citations
                                        %   correctly because it was included with the
                                        %   previous LaTeX template.
}

%%%%%%%%%%%%%%%%%%%%%%%%%%%%%%%%%%%%%%%
% Tables and figures
%%%%%%%%%%%%%%%%%%%%%%%%%%%%%%%%%%%%%%%
\captiondelim{. }                       %~Use period (.) after caption number instead of
                                        %   colon (:). Change according to style guide.
%% \captionstyle[\raggedright]%         % Set justifcation for [one line captions]
%%   {\raggedright}                     %   and {multiple line captions}
\setlength{\belowcaptionskip}{0pt}      % Bring caption down closer to figure/table
\makeatletter                           % Consecutive numbering throughout
  \counterwithout{figure}{chapter}      %   (including back matter)
  \counterwithout{table}{chapter}
  \renewcommand\@memfront@floats{}
  \renewcommand\@memmain@floats{}
  \renewcommand\@memback@floats{}
\makeatletter

%%% Tables %%%
%
% Note: 'memoir' natively supports commands from the following table-related packages:
%   tabularx, ccaption, booktabs.
% Everyone has particular ideas about how tables should look, so you may need to
%   load additional packages and modify the code below to get tables (and figures) to
%   look the way you want them to.
\setfloatadjustment{table}{\raggedright}% Left justify material inside table floats
\usepackage{tabu}                       % 'tabu' is an excellent table package; it can
                                        %   automatically size column widths and has a
                                        %   lot of customizations that other packages do
                                        %   not. It also has a 'longtabu' environment that
                                        %   emulates 'longtable' with additional features
                                        %   from the 'tabu' package. If you don't want
                                        %   to use it, you can comment this line out.
\BeforeBeginEnvironment{table}%         % Single space inside table environment
  {\SingleSpacing}
\AfterEndEnvironment{table}
  {\DoubleSpacing}

%%% Figures %%%
\setfloatadjustment{figure}%            % Left justify material inside figure floats
  {\raggedright}
\BeforeBeginEnvironment{figure}%        % Single space inside figure environment
  {\SingleSpacing}
\AfterEndEnvironment{figure}
  {\DoubleSpacing}

\makeatletter                           % Define custom macro called '\maxwidth{}' that
  \def\maxwidth#1{%                     %   allows you to specify the maximum width of an
    \ifdim%                             %   imported image. See below for an example.
      \Gin@nat@width>#1 #1%             %
    \else%                              % Source: http://tex.stackexchange.com/questions/86350/includegraphics-maximum-width
      \Gin@nat@width%
    \fi}
\makeatother
%
% Example \maxwidth:
%
%   \includegraphics[width=\maxwidth]{\textwidth}]{image.pdf}
%
% Note: This will keep an image inside the horizontal margins assuming the image starts
%   on the right margin (i.e., no horizontal space before the image).

%%%%%%%%%%%%%%%%%%%%%%%%%%%%%%%%%%%%%%%
% Hyperref settings
%%%%%%%%%%%%%%%%%%%%%%%%%%%%%%%%%%%%%%%

%%% URL Settings %%%
\PassOptionsToPackage{hyphens}{url}
\usepackage[breaklinks=true]{hyperref}  % 'hyperref' should be loaded at the end of the
                                        %   preamble; Note: the uppercasing commands used
                                        %   throughout the preamble can conflict with it,
                                        %   especially when non-standard fonts or
                                        %   different file encodings are used
\urlstyle{same}                         % Set URLs in the same font as regular text

\tolerance 1414                         % Help URLs from entering margins
\hbadness 1414                          %   Source: https://tex.stackexchange.com/questions/3033/forcing-linebreaks-in-url
\emergencystretch 1.5em
\hfuzz 0.3pt
\widowpenalty=10000
\vfuzz \hfuzz

%%% Create metadata strings
\usepackage{hyperxmp}                   % For metadata
\renewcommand*{\do}[1]{#1\ }%           % Build title string to output to pdf document
\newcommand*{\onelinetitle}{%
  \dolistloop{\titlelines}%
}
\edef\theonelinetitle%
  {\onelinetitle}

\renewcommand*{\do}[1]{{#1}\ }%         % Build keyword string to output to pdf document
\newcommand*{\pdfkeywordsstring}{%
  \dolistloop{\keywords}%
}
\edef\thepdfkeywordsstring%
  {\pdfkeywordsstring}

\newcommand*{\pdfcopyrightstring}%      % Build copyright message string
  {Copyright \copyright\space\gradyear\ by \Author.%
  {\space}All rights reserved.}

\ifpdf                                  % Build pdf creator string (for pdfTeX)
  \makeatletter
  \def\extractpdftexversion#1-#2-#3 #4%
    \@nil{#3}
  \edef\pdfcreator{pdfTeX \expandafter%
    \extractpdftexversion\pdftexbanner\@nil}
  \makeatother
\fi
\ifxetex                                % Build pdf creator string (for XeTeX)
  \edef\pdfcreator{XeTeX %
    \the\XeTeXversion\XeTeXrevision}
\fi

\edef\pdfsummary{%                      % Build pdf summary
  A \documentname Presented in\space
  Partial Fulfillment of the\space
  Requirements for a \degreename\space
  from Arizona State University}

%%% Enter metadata and other settings
\hypersetup{                            % Set pdf metadata
  pdftitle={\theonelinetitle},          % Title
  pdfauthor={\Author},                  % Author
  pdfcreator={\pdfcreator},             % Enter the TeX writer for good documentation
 %pdfproducer={},                       % Let 'pdfproducer' be filled automatically
  pdfsubject={\pdfsummary},             % Subject of the document
  pdfkeywords=\thepdfkeywordsstring,    % List of keywords
  hidelinks={true},                     % Links look like regular text (no colors, boxes)
  breaklinks={true},                    % Allow links to break across lines
}
\ifxetex                                % If processing with XeTeX
  \hypersetup{unicode=true}             % Must use 'true' in XeTeX
\else
  \hypersetup{unicode=true}             % Default is to use 'true' otherwise, as well
\fi
\ifpdf                                  % Copyright message; probably only works in pdfTeX
  \hypersetup{
    pdfcopyright={\pdfcopyrightstring},
    pdfinfo={%
      Copyright=\pdfcopyrightstring%
    }%
  }
\fi

\usepackage%
  [numbered,%                           % Include numbers of sections in bookmarks
  open%                                 % Bookmark tree already expanded when PDF opened
  ]%
  {bookmark}
\bookmark[page=1,rellevel=0,%           % Create bookmark of title page at root level
  keeplevel=true]{Title Page}
\preto{\tableofcontents}{%              % Create bookmark for TOC
  \hypertarget{tocpage}{}%
  \bookmark[dest=tocpage,rellevel=0,%
    keeplevel=true]{\contentsname}%
}

%%%%%%%%%%%%%%%%%%%%%%%%%%%%%%%%%%%%%%%
% Copyright page
%%%%%%%%%%%%%%%%%%%%%%%%%%%%%%%%%%%%%%%
\newcommand{\copyrightpageASU}{%        % Create copyright page
  \thispagestyle{empty}
  \titlepagesetup
  ~\\ \vfill
    \begin{center}
      \copyright\space\gradyear\space%
      \Author\\%
      All Rights Reserved%
    \end{center}%
  \clearpage%
  \closetitlepagesetup
}

%%%%%%%%%%%%%%%%%%%%%%%%%%%%%%%%%%%%%%%
% Sample settings
%%%%%%%%%%%%%%%%%%%%%%%%%%%%%%%%%%%%%%%
\providetoggle{sample}                  % True = demonstration of template
\settoggle{sample}{false}
\iftoggle{sample}{%
  \newcounter{tablecounter}
  \setcounter{tablecounter}{1}
  \newcounter{figurecounter}
  \setcounter{figurecounter}{1}
}{%
}

%%%%%%%%%%%%%%%%%%%%%%%%%%%%%%%%%%%%%%%
% Debugging Help
%%%%%%%%%%%%%%%%%%%%%%%%%%%%%%%%%%%%%%%
\usepackage{lipsum}                     % Outputs dummy text

%%%%%%%%%%%%%%%%%%%%%%%%%%%%%%%%%%%%%%%%%%%%%%%%%%%%%%%%%%%%%%%%%%%%%%%%%%%%%%
% Document
%%%%%%%%%%%%%%%%%%%%%%%%%%%%%%%%%%%%%%%%%%%%%%%%%%%%%%%%%%%%%%%%%%%%%%%%%%%%%%
\begin{document}

%%%%%%%%%%%%%%%%%%%%%%%%%%%%%%%%%%%%%%%
% Title page
%%%%%%%%%%%%%%%%%%%%%%%%%%%%%%%%%%%%%%%
\titlepageASU

%%%%%%%%%%%%%%%%%%%%%%%%%%%%%%%%%%%%%%%
% Copyright page
%%%%%%%%%%%%%%%%%%%%%%%%%%%%%%%%%%%%%%%
\copyrightpageASU                       %~If you don't want to have a copyright page,
                                        %   comment out this line

%%%%%%%%%%%%%%%%%%%%%%%%%%%%%%%%%%%%%%%
% Front matter
%%%%%%%%%%%%%%%%%%%%%%%%%%%%%%%%%%%%%%%
\chapterstyle{ASU}
\pagestyle{ASU}
\frontmatter

\chapter*{Abstract}                     % Abstract is required
Reverse engineering is critical to reasoning about how a system behaves.
While complete access to a system inherently allows for perfect analysis, partial access is inherently uncertain.
This is the case for an individual agent in a distributed system.
Inductive Reverse Engineering (IRE) enables analysis under such circumstances.

IRE does this by producing program spaces consistent with individual input-output examples for a given domain-specific language.
Then, IRE intersects those program spaces to produce a generalized program consistent with all examples.
IRE, an easy to use framework, allows this domain-specific language to be specified in the form of $Theorist$s, which produce $Theory$s, a succinct way of representing the program space.

Programs are often much more complex than simple string transformations.
One of the ways in which they are more complex is in the way that they follow a conversation-like behavior, potentially following some underlying protocol.
As a result, IRE represents program interactions as $Conversation$s in order to more correctly model a distributed system.
This, for instance, enables IRE to model dynamically captured inputs received from other agents in the distributed system.

While domain-specific knowledge provided by a user is extremely valuable, such information is not always possible.
IRE mitigates this by automatically inferring program grammars, allowing it to still perform efficient searches of the program space.
It does this by intersecting conversations prior to synthesis in order to understand what portions of conversations are constant.

IRE exists to be a tool that can aid in automatic reverse engineering across numerous domains.
Further, IRE aspires to be a centralized location and interface for implementing program synthesis and automatic black box analysis techniques.
                        %<Enter the name of the .tex file containing your
                                        %   your abstract or omit this line and type in
                                        %   your abstract here.

\chapter*{Dedication}                   %~Dedication is optional
%\clearpage                             %~If you don't wish to display the heading
                                        %   'Dedication', comment out the previous line
                                        %   and use this one instead.
\leavevmode\vfill
Thank you to Adam, for not expelling me when I reported a potential security vulnerability in his principles of programming languages' project submission server, but instead inviting me to take his course on software security and inviting me to join ASU's hacking team, the pwndevils.
His course immediately hooked me into the "cybersecurity" world.
It was more game-like than it was course-like, with an actual scoreboard on several of the assignments, which ultimately created the passion I have today for solving security challenges.
From there, Adam gave me the amazing opportunity to do security research with him and serve as a teaching assistant in his software security course.
These experiences have been truly invaluable and absolutely amazing to be a part of.

Thank you to Yan, for showing me that anything can be done the day before it is absolutely necessary, as we worked through the night and early morning at a coffee shop--just before his first class on systems security--to create a challenge infrastructure (which would be hacked by one of the students just hours later).
I remember listening to Yan's talk on angr at DEF CON 23, a couple of years before he joined our lab--and I remember how excited I was when I learned that this legendary hacker might join our lab.
The level of energy and humor that Yan has brought to the lab is truly inspiring.

It has been an absolute blast working with both Adam and Yan, as well as Fish, Tiffany, and the rest of the lab. Our amazing lab is the reason that I continue to do research.
\vfill
                      %<Enter the name of the .tex file containing your
                                        %   your dedication or omit this line and type in
                                        %   your dedication here.

%% \chapter*{Acknowledgments}              %~Acknowledgments are optional
%% Lorem ipsum dolor sit amet, consectetuer adipiscing elit. Ut purus elit, vestibulum ut,
placerat ac, adipiscing vitae, felis. Curabitur dictum gravida mauris. Nam arcu libero,
nonummy eget, consectetuer id, vulputate a, magna. Donec vehicula augue eu neque.
Pellentesque habitant morbi tristique senectus et netus et malesuada fames ac turpis
egestas. Mauris ut leo. Cras viverra metus rhoncus sem. Nulla et lectus vestibulum urna
fringilla ultrices. Phasellus eu tellus sit amet tortor gravida placerat. Integer sapien
est, iaculis in, pretium quis, viverra ac, nunc. Praesent eget sem vel leo ultrices
biben- dum. Aenean faucibus. Morbi dolor nulla, malesuada eu, pulvinar at, mollis ac,
nulla. Curabitur auctor semper nulla. Donec varius orci eget risus. Duis nibh mi,
congue eu, accumsan eleifend, sagittis quis, diam. Duis eget orci sit amet orci
dignissim rutrum.                 %<Enter the name of the .tex file containing your
%%                                         %   acknowledgments or omit this line and type in
%%                                         %   your acknowledgments here.

\iftoggle{usemicrotype}                 % If 'microtype' is in use, turn off protrusion
  {\microtypesetup{protrusion=false}}%  %   for TOC
  {}
\clearpage                              % Output table of contents on a new page
\contentslistsetup                      % Change page layout for contents lists

\pagestyle{ASUtoc}
\tableofcontents*                       % Starred version leaves TOC heading out of TOC
\addtocontents{toc}%                    % List of ... needs to be on left margin, but
  {\setlength{\cftchapterindent}%       %   they inherit 'chapter' formatting, so override
    {0em}%
  }

%% \clearpage
%% \pagestyle{ASUlot}
%% \listoftables                           % List of Tables should appear in TOC, so use
%%                                         %   unstarred version of \listoftables
\clearpage
\pagestyle{ASUlof}
\listoffigures                          % List of Figures should appear in TOC, so use
                                        %   unstarred version of \listoffigures

\phantomsection                         % \phantomsection is needed before using
                                        %   \addtocontents when it contains certain macros
                                        %   when also using 'hyperref' package
\addtocontents{toc}%                    % Undo manual override above for chapter indent,
  {\setlength{\cftchapterindent}%       %   so actual chapters in the TOC are indented
    {\levelindentincrement}%            %   correctly
  }
\setlength{\afterchapskip}%             % Set vertical space between chapter title and
  {\baselineskip}                       %   first paragraph; equivalent to two line breaks

\phantomsection
\addcontentsline{toc}{part}{Chapter}    % Add "Chapter" to TOC here at 'part' level
\phantomsection
\addtocontents{toc}%                    % Add this 'mark' to TOC so subsequent pages use
  {\protect\markboth{CHAPTER}{Page}}    %   the "CHAPTER" heading

\iftoggle{usemicrotype}                 % If 'microtype' is in use, turn protrusion back
  {\microtypesetup{protrusion=true}}%   %   on
  {}

\clearpage                              % Note: All these changes have to be above a
                                        %   a '\clearpage' before '\mainmatter'

\pagestyle{ASU}                         % Switch back to regular page style for remainder
                                        %   of the document
\closecontentslistsetup                 % Undo page layout for contents lists

%\chapter{Definitions}                   %~OTHER LISTS (optional)
%\input{definitions}                     %<Enter the name of the .tex file or omit this
                                        %   line and type in here.

%\chapter{Preface}                       %~PREFACE (optional, less than 10 pages)
%\input{preface}                         %<Enter the name of the .tex file or omit this
                                        %   line and type in here.

%%%%%%%%%%%%%%%%%%%%%%%%%%%%%%%%%%%%%%%
% Body
%%%%%%%%%%%%%%%%%%%%%%%%%%%%%%%%%%%%%%%
\mainmatter
\chapter{Introduction}

Reverse engineering is a methodology for precisely analyzing the internal workings and substructure of a process or system in order to better understand how it works.
In practice, it is done in the absence of high-level specifications and can be thought of as working backward through the standard engineering process---design towards implementation---and instead, implementation towards design.
In theory, reverse engineering is relatively straightforward: simply observe how the internals are operating and how the subcomponents are connected.
Of course, it is more nuanced than this; but even so, a complete working system that can be observed is, by its very nature, perfectly descriptive of what it does and how it works.
This is an underlying requirement of standard reverse engineering.
In hardware, a physical object is disassembled and examined.
In software, its source code is read through, or in its absence, binary disassembled and machine code analyzed.

Consider, however, the task of reverse engineering without complete access to observing the system.
This is the case in distributed systems, where some agent has only partial access to the overall system.
Take for instance web applications, where a client makes a request to a server and is only made aware of its response.
From the client's perspective, the server is merely a black-box---an oracle---that takes some input and returns some output.
What happens in between is left unknown to the client.
In such cases, reverse engineering becomes inherently uncertain.

Further consider the problem of systems in which interactions take place between persons and computers; for instance, a human interacting with some computer program in a repetitive way.
This constitutes a distributed system, where part of the program takes place in the computer, but also part of it takes place in the person's intentions towards interacting with the computer.
In such cases, the program occurring in the person's intentions---in the person's mind---is but a black-box to the computer.
It is in this way that reverse engineering may be applied not only towards analyzing a computer, but also a person.
Reverse engineering may be useful here in order to profile or improve upon the user experience of the person.

Although reverse engineering in these situations becomes an inherently uncertain process, this does not stop human-efforts in reverse engineering.
In practice, humans build up an entire model of the black-box system under analysis.
They make assumptions based on past systems, attempt to rule out and confirm these assumptions, and use intuition as a means of guiding this process.
This inductive reasoning forms the basis for inductive reverse engineering: IRE.

IRE serves to solve this problem of reverse engineering in a black-box environment.
IRE is an easy to use, open source, Python 3 framework, that enables users to transform input-output examples into executable programs consistent with those examples.
This effectively allows for programming by example.
Users can easily introduce domain-specific knowledge about the problem they are working on to further enhance this process.

\chapter{Background}

This sort of reverse engineering without access to internals has become a massively important skill, and in particular, critical to cybersecurity.
Take for instance phone phreaking, where early hackers mapped out the phone network and how it worked simply by interacting with it using various tones and observing the results \cite{mitnick2011art}.
In more recent times, penetration testing has become an important profession that often relies on reverse engineering in order to audit the security of companies from the perspective of an outside attacker.

One of the common tasks of a penetration tester is in utilizing tools such as black-box web vulnerability scanners.
These scanners naively scan a web application, blindly sending exploits at input points and trying to detect if they worked.
A survey of these tools shows that there is much work to be done in improving them \cite{doupe2010johnny}.
Furthermore, efforts to provide more semantic information about an application's state has proven to be effective, despite still fundamentally only having black-box access \cite{doupe2012enemy}.

In response to the demand for reverse engineering, recent efforts have been made to push towards Cyber Reasoning Systems which aid in this effort, and in some cases entirely automate it.
The Defense Advanced Research Projects Agency (DARPA) led an initiative to develop fully autonomous systems capable of reverse engineering and exploiting challenge binaries in their Cyber Grand Challenge \cite{shellphish2017cyber}.
This has led to significant advances in program analysis and various techniques surrounding state-of-the-art reverse engineering.
Many of these techniques can be seen in open source frameworks for performing program analysis including \textit{angr} and \textit{Manticore} \cite{shoshitaishvili2016state, stephens2016driller, shoshitaishvili2015firmalice}.
These frameworks provide users with tools for precisely reasoning about a program by analyzing their internals.

\chapter{Theorists and Theories}

IRE works around two central primitives: theorists and theories.
Theorists use input-output examples to produce theories.
They act as a sort of domain-specific language.
Theories use input to produce output.
They model the synthesized program.

\image{theorist.png}{Theorist}{0.7}

\image{theory.png}{Theory}{0.7}

\section{Grammar Theorists}

IRE fundamentally solves a parsing problem.
It must parse input-output examples to produce a program, not entirely dissimilar to a programming language parser which parses source code to produce a program.
For this reason, it is useful to think in terms of context-free grammars.
While in the programming language realm context-free grammars strive to be unambiguous, IRE leverages ambiguity to produce the program space.
Consider, for instance, the HTML grammar theorist (Figure \ref{fig:html_grammar}) written using the IRE framework.

\begin{figure}[tb]
\begin{pythoncode}
import ire

class HTML(ire.GrammarTheorist, entry='html'):
    html = ire.RepeatTheorist(html_element) ~$\label{html_theorist}$~
    html_element = (open_tag & close_tag & self_tag) | data ~$\label{html_element_theorist}$~

    open_tag = '<' + tag_content + '>'
    close_tag = '</' + tag_name + '>'
    self_tag = '<' + tag_content + '/>'

    data = '.|\n' & ire.FunctionTheorist(None) & ire.FunctionTheorist(b64) ~$\label{data_theorist}$~

    tag_content = (tag_name + whitespace + attributes) | tag_name
    tag_name = '[a-zA-Z]+'

    attributes = (attribute + whitespace + attributes) | attribute
    attribute = '[a-zA-Z]+' & ('[a-zA-Z]+="' + attribute_value + '"')
    attribute_value = '[a-zA-Z0-9]'

    whitespace = '\s+'
\end{pythoncode}
\caption{HTML Grammar Theorist}
\label{fig:html_grammar}
\end{figure}

Here, a $GrammarTheorist$ is defined, with $html$ being the entry point, or start symbol in context-free grammar terminology.
Variables that appear on the left side of an $=$ indicate a nonterminal symbol, while all other expressions are terminal symbols.
Each of these assignments forms the basis for a production rule.

Strings are implicitly $RegexTheorist$s, which as the name suggests, perform regex matching.
Addition expressions (using the $+$ operator) are implicitly $ConcatTheorist$s, which serve to concatenate theorists, and their produced theories, together.

Rather than defining several production rules for the same nonterminal, operators $\&$ and $|$ both respectively serve to implicitly create $AndTheorist$s and $OrTheorist$s.
All theorists within an $AndTheorist$ will attempt to produce theories.
After some theorist within an $OrTheorist$ has produced some number of theories, any remaining theorists will not be given a chance to produce theories.
This evaluation takes place from left to right and allows for a conditionally restricted search space, and consequently more efficient parsing.
On line \ref{html_element_theorist}, the $OrTheorist$ indicates that an $html\_element$ can be an $open\_tag$, $close\_tag$ and $self\_tag$, or if that is not the case then it must be $data$.

On line \ref{html_theorist}, the $RepeatTheorist$ indicates that parsing should repeatedly consume (one or more times) $html\_element$s until it no longer can.
Implementation wise, this is more efficient than how looping is traditionally performed in context-free grammars by having a self-referential production rule.

$FunctionTheorist$s wrap functions to make them behave as theorists.
Line \ref{data_theorist} showcases two of these theorists.
$FunctionTheorist(None)$ indicates the identity function ($f(x)=x$).
$FunctionTheorist(b64)$ on the other hand assumes the existence of a $b64$ function (defined elsewhere), which base64 encodes its input.


\section{Theorists to Theories}

Consider the simple HTML echo program (Figure \ref{fig:simple_html_echo_program}) in order to understand how IRE is able to utilize the prior HTML grammar theorist (Figure \ref{fig:html_grammar}) to reason about this program and produce theories.

\begin{figure}[tb]
\begin{pythoncode}
def echo(name, msg):
    return \
f"""
<html>
    <head>
        <title>Echo</title>
    </head>
    <body>
        <p>Hello {name}!</p>
        <p>"{msg}" is {b64(msg)}</p>
    </body>
</html>
"""
\end{pythoncode}
\caption{Simple HTML Echo Program}
\label{fig:simple_html_echo_program}
\end{figure}

This simple program (Figure \ref{fig:simple_html_echo_program}) transforms the inputs into an HTML output.
Inputs \mintinline{python}{name='Paul'} and \mintinline{python}{msg='Hello World'} result in the output shown in Figure \ref{fig:output_of_simple_html_echo_program}.

\begin{figure}[tb]
\begin{data}
\let =\enskip
<html>
    <head>
        <title>Echo</title>
    </head>
    <body>
        <p>Hello Paul!</p>
        <p>"Hello World" is SGVsbG8gV29ybGQ=</p>
    </body>
</html>
\end{data}
\caption{Output of Simple HTML Echo Program}
\label{fig:output_of_simple_html_echo_program}
\end{figure}

Running this input-output example (Figure \ref{fig:output_of_simple_html_echo_program}) through the prior HTML grammar theorist (Figure \ref{fig:html_grammar}) results in the theory shown in Figure \ref{fig:simple_html_echo_theory}, displayed as a simple program summary using the IRE framework.

\begin{figure}[tb]
\begin{data}
\let =\enskip
\textcolor{gray}{\{}
<html>
    <head>
        <title>Echo</title>
    </head>
    <body>
        <p>Hello \textcolor{gray}{\{}\textcolor{blue}{\{\{Input[0]\}\}}\textcolor{gray}{, }\textcolor{blue}{Paul}\textcolor{gray}{\}}!</p>
        <p>"\textcolor{gray}{\{}\textcolor{blue}{\{\{Input[1]\}\}}\textcolor{gray}{, }\textcolor{blue}{Hello World}\textcolor{gray}{\}}" is \textcolor{gray}{\{}\textcolor{blue}{\{\{b64(Input[1])\}\}}\textcolor{gray}{, }\textcolor{blue}{SGVsbG8gV29ybGQ=}\textcolor{gray}{\}}</p>
    </body>
</html>
\textcolor{gray}{\}}
\end{data}
\caption{Simple HTML Echo Theory}
\label{fig:simple_html_echo_theory}
\end{figure}

In this simple program summary (Figure \ref{fig:simple_html_echo_theory}), the gray curly braces and commas represent $UnionTheory$s, and the blue text represents individual theories within those $UnionTheory$s.
$UnionTheory$s are a way of succinctly representing the program space, allowing common theories among candidate program traces to not be repeated, effectively forming a directed acyclic graph.
Double curly braces indicate theories that depend on the input (e.g. $FunctionTheory$s).

This shows an interesting result common to running a theorist against only one input-output example: the complete original constant output by itself appears as a possible program trace within the program space.
These constant theories can be ruled out only with more input-output examples.

\section{Theory Intersection}
\label{sec:theory_intersection}

In order to collapse the program space, theories resulting from different input-output examples must be intersected together.

\image{intersection.png}{Theory Intersection}{0.7}

Inputs \mintinline{python}{name='Pablo'} and \mintinline{python}{msg='Hola Mundo'} result in the output shown in Figure \ref{fig:another_output_of_simple_html_echo_program}.

\begin{figure}[tb]
\begin{data}
\let =\enskip
<html>
    <head>
        <title>Echo</title>
    </head>
    <body>
        <p>Hello Pablo!</p>
        <p>"Hola Mundo" is SG9sYSBNdW5kbw==</p>
    </body>
</html>
\end{data}
\caption{Another Output of Simple HTML Echo Program}
\label{fig:another_output_of_simple_html_echo_program}
\end{figure}

Running this input-output example (Figure \ref{fig:another_output_of_simple_html_echo_program}) through the prior HTML grammar theorist (Figure \ref{fig:html_grammar}) results in the theory shown in Figure \ref{fig:another_simple_html_echo_theory}, displayed as a simple program summary using the IRE framework.

\begin{figure}[tb]
\begin{data}
\let =\enskip
\textcolor{gray}{\{}
<html>
    <head>
        <title>Echo</title>
    </head>
    <body>
        <p>Hello \textcolor{gray}{\{}\textcolor{blue}{\{\{Input[0]\}\}}\textcolor{gray}{, }\textcolor{blue}{Pablo}\textcolor{gray}{\}}!</p>
        <p>"\textcolor{gray}{\{}\textcolor{blue}{\{\{Input[1]\}\}}\textcolor{gray}{, }\textcolor{blue}{Hola Mundo}\textcolor{gray}{\}}" is \textcolor{gray}{\{}\textcolor{blue}{\{\{b64(Input[1])\}\}}\textcolor{gray}{, }\textcolor{blue}{SG9sYSBNdW5kbw==}\textcolor{gray}{\}}</p>
    </body>
</html>
\textcolor{gray}{\}}
\end{data}
\caption{Another Simple HTML Echo Theory}
\label{fig:another_simple_html_echo_theory}
\end{figure}

Intersecting the first theory (Figure \ref{fig:simple_html_echo_theory}) with the second theory (Figure \ref{fig:another_simple_html_echo_theory}), results in just one program trace in the program space: our original program (Figure \ref{fig:simple_html_echo_program}).

\begin{figure}[tb]
\begin{data}
\let =\enskip
\textcolor{gray}{\{}
<html>
    <head>
        <title>Echo</title>
    </head>
    <body>
        <p>Hello \textcolor{gray}{\{}\textcolor{blue}{\{\{Input[0]\}\}}\textcolor{gray}{\}}!</p>
        <p>"\textcolor{gray}{\{}\textcolor{blue}{\{\{Input[1]\}\}}\textcolor{gray}{\}}" is \textcolor{gray}{\{}\textcolor{blue}{\{\{b64(Input[1])\}\}}\textcolor{gray}{\}}</p>
    </body>
</html>
\textcolor{gray}{\}}
\end{data}
\caption{Simple HTML Echo Theory Intersected}
\label{fig:simple_html_echo_theory_intersected}
\end{figure}

\chapter{Conversations}

Often times, distributed programs follow a conversation-like behavior, potentially following some underlying protocol.
In such cases, program synthesis cannot merely take place over only simple input-output examples.
Instead, input-output examples must be generalized to conversation examples.
Here, discerning between input and output doesn't necessarily make sense, as it depends on an agent's perspective.
Instead, a conversation takes place over a series of messages, where a message has some source agent and destination agent.
Further, these agents have context--a generalization of input--representing what an agent individually brings as their input to a particular conversation.
This allows for some notion of state.
Nevertheless, in a deterministic program, identical contexts will produce identical conversations.

\image{conversation.png}{Conversation}{0.5}

\section{Capturing Theorists and Theories}

An agent is not aware of other agents' contexts; only its own context and any received messages.
This is a fundamental problem in reverse engineering distributed systems.
Therefore, IRE must use an agent's received messages in order to recover the sender's context.
It does so using $CaptureTheorist$s and $CaptureTheory$s.

$CaptureTheorist$s wrap other theorists and effectively leverage their ability to parse.
Those $CaptureTheorist$s go on to create $CaptureTheory$s, which also wrap those same theorists.
During synthesis, the underlying theorists attempt to produce theories, and anything consumed during this is captured.
This captured data may then be used in later theorists, much like a traditional input.
Those produced theories do not actually become a part of the program space, but instead the $CaptureTheory$s storing the underlying theorist does.
This allows this parsing done during the synthesis to be performed again during execution of the theory.

\image{capture.png}{Capture Theorist and Capture Theory}{0.7}

Consider the following HTTP theorists (Figure \ref{fig:http_theorist}) written using the IRE framework to understand how these $CaptureTheorist$s and $CaptureTheory$s may be applied.

\begin{figure}[H]
\begin{pythoncode}
class HTTPRequest(ire.OutputMessageTheorist, entry='request'):
    request = request_prolog + headers + '\n' + request_contents
    request_prolog = '(GET|POST) ' + request_url + ' HTTP/1.1\n'
    request_url = ire.RepeatTheorist(request_url_data)
    request_url_data = '[^ ]' & inputs

    headers = ire.RepeatTheorist(header)
    header = '[a-zA-Z-]+: ' + ire.RepeatTheorist(header_data) + '\n'
    header_data = '.' & inputs

    request_contents = ire.RepeatTheorist(request_contents_data) | ''
    request_contents_data = '.|\n' & inputs

    inputs = ire.FunctionTheorist(None) & ire.FunctionTheorist(b64)


class HTTPResponse(ire.InputMessageTheorist, entry='response'):
    response = response_prolog + headers + '\n' + response_contents
    response_prolog = 'HTTP/1.1 [0-9]+ [a-zA-Z ]+\n'

    headers = ire.RepeatTheorist(cookie_header | header)
    cookie_header = 'Set-Cookie: ' + cookie + '\n'
    header = '[a-zA-Z-]+: ' + ire.RepeatTheorist(header_data) + '\n'
    header_data = '.' & inputs

    response_contents = HTML() | ''

    cookie = ire.CaptureTheorist(cookie_data, 'cookie') ~$\label{cookie_theorist}$~
    cookie_data = '[a-zA-Z]+=[a-zA-Z0-9]+'

    inputs = ire.FunctionTheorist(None) & ire.FunctionTheorist(b64)


http = HTTPRequest() & HTTPResponse()
\end{pythoncode}
\caption{HTTP Theorist}
\label{fig:http_theorist}
\end{figure}

In the case of web applications, it is common for an HTTP server to negotiate some token, commonly known as a cookie, for keeping track of state with its clients \cite{barth2011rfc}.
Cookies are a way of enabling statefulness in the otherwise stateless HTTP protocol.
An HTTP client will include the cookies associated with a particular server with all web requests made to that server.
By introducing a $CaptureTheorist$ on line \ref{cookie_theorist}, this behavior is effectively conveyed and enables IRE to capture the cookie.

Now consider the following HTTP conversation (Figure \ref{fig:simple_http_conversation}) that results from the client's context of \mintinline{python}{username='Paul'}, \mintinline{python}{password='p455w0rd'}, and \mintinline{python}{msg='Hello_World'}; and unknown to the client, server's context of \mintinline{python}{cookie='sessionid=12345'}.

\begin{figure}[H]
\begin{data}
\let =\enskip
\begin{data}
POST /login HTTP/1.1
Host: example.com
Content-Type: application/json
\medskip
\{"username": "Paul", "password": "p455w0rd"\}
\end{data}

\begin{data}
HTTP/1.1 200 OK
Set-Cookie: sessionid=12345
Connection: close
\medskip
\end{data}

\begin{data}
GET /echo?msg=Hello\_World HTTP/1.1
Host: example.com
Cookie: sessionid=12345
\medskip
\end{data}

\begin{data}
HTTP/1.1 200 OK
Content-Type: text/html; charset=UTF-8
Connection: close
\medskip
<html>
    <head>
        <title>Echo base64</title>
    </head>
    <body>
        <p>Hello Paul!</p>
        <p>"Hello\_World" is SGVsbG9fV29ybGQ=</p>
    </body>
</html>
\end{data}
\end{data}
\caption{Simple HTTP Conversation}
\label{fig:simple_http_conversation}
\end{figure}

Running this conversation example (Figure \ref{fig:simple_http_conversation}) through the prior HTTP theorist (Figure \ref{fig:http_theorist}) results in the following theory (Figure \ref{fig:simple_http_theory}), displayed as a simple program summary using the IRE framework.

\begin{figure}[H]
\begin{data}
\let =\enskip
\begin{data}
\textcolor{gray}{\{}
POST /login HTTP/1.1
Host: example.com
Content-Type: application/json
\medskip
\{"username": "\textcolor{gray}{\{}\textcolor{blue}{\{\{Input[username]\}\}}\textcolor{gray}{, }\textcolor{blue}{Paul}\textcolor{gray}{\}}", "password": "\textcolor{gray}{\{}\textcolor{blue}{\{\{Input[password]\}\}}\textcolor{gray}{, }\textcolor{blue}{p455w0rd}\textcolor{gray}{\}}"\}
\textcolor{gray}{\}}
\end{data}

\begin{data}
\textcolor{gray}{\{}
HTTP/1.1 200 OK
Set-Cookie: \textcolor{gray}{\{}\textcolor{blue}{\{\{Capture[cookie]\}\}}\textcolor{gray}{\}}
Connection: close
\medskip
\textcolor{gray}{\}}
\end{data}

\begin{data}
\textcolor{gray}{\{}
GET /echo?msg=\textcolor{gray}{\{}\textcolor{blue}{\{\{Input[msg]\}\}}\textcolor{gray}{, }\textcolor{blue}{Hello\_World}\textcolor{gray}{\}} HTTP/1.1
Host: example.com
Cookie: \textcolor{gray}{\{}\textcolor{blue}{\{\{Input[cookie]\}\}}\textcolor{gray}{, }\textcolor{blue}{sessionid=12345}\textcolor{gray}{\}}
\medskip
\textcolor{gray}{\}}
\end{data}

\begin{data}
\textcolor{gray}{\{}
HTTP/1.1 200 OK
Content-Type: text/html; charset=UTF-8
Connection: close
\medskip
<html>
    <head>
        <title>Echo base64</title>
    </head>
    <body>
        <p>Hello \textcolor{gray}{\{}\textcolor{blue}{\{\{Input[username]\}\}}\textcolor{gray}{, }\textcolor{blue}{Paul}\textcolor{gray}{\}}!</p>
        <p>"\textcolor{gray}{\{}\textcolor{blue}{\{\{Input[msg]\}\}}\textcolor{gray}{, }\textcolor{blue}{Hello\_World}\textcolor{gray}{\}}" is \textcolor{gray}{\{}\textcolor{blue}{\{\{b64(Input[username])\}\}}\textcolor{gray}{, }\textcolor{blue}{SGVsbG9fV29ybGQ=}\textcolor{gray}{\}}</p>
    </body>
</html>
\textcolor{gray}{\}}
\end{data}
\end{data}
\caption{Simple HTTP Theory}
\label{fig:simple_http_theory}
\end{figure}

With another example conversation, this program space could be collapsed as discussed in Section \ref{sec:theory_intersection}.

\chapter{Grammarless Synthesis}

While it is useful to be able to provide domain-specific knowledge to an analysis, doing so is not always possible.
Providing a grammar to IRE allows it to be more efficient in its search, by reducing the possible program space.
Consider, for instance, the grammar shown in Figure \ref{fig:generic_theorist} which is highly generic.

\begin{figure}[tb]
\begin{pythoncode}
class Generic(ire.GrammarTheorist, entry='entry'):
    entry = ire.RepeatTheorist(data)
    data = '.|\n' & ire.FunctionTheorist(None) & ire.FunctionTheorist(b64)
\end{pythoncode}
\caption{Generic Theorist}
\label{fig:generic_theorist}
\end{figure}

This grammar (Figure \ref{fig:generic_theorist}) repeatedly matches against a single byte and derivations of the input.
Repeatedly matching against a single byte is what effectively drives the parsing forward.

Inputs \mintinline{python}{name='Paul'} and \mintinline{python}{msg='body'} run against the prior simple program (Figure \ref{fig:simple_html_echo_program}) and through the generic theorist (Figure \ref{fig:generic_theorist}) result in the theory shown in Figure \ref{fig:generic_theory}, displayed as a simple program summary using the IRE framework.

\begin{figure}[tb]
\begin{data}
\let =\enskip
\textcolor{gray}{\{}
<html>
    <head>
        <title>Echo</title>
    </head>
    <\textcolor{gray}{\{}\textcolor{blue}{\{\{Input[1]\}\}}\textcolor{gray}{, }\textcolor{blue}{body}\textcolor{gray}{\}}>
        <p>Hello \textcolor{gray}{\{}\textcolor{blue}{\{\{Input[0]\}\}}\textcolor{gray}{, }\textcolor{blue}{Paul}\textcolor{gray}{\}}!</p>
        <p>"\textcolor{gray}{\{}\textcolor{blue}{\{\{Input[1]\}\}}\textcolor{gray}{, }\textcolor{blue}{body}\textcolor{gray}{\}}" is \textcolor{gray}{\{}\textcolor{blue}{\{\{b64(Input[1])\}\}}\textcolor{gray}{, }\textcolor{blue}{Ym9keQ==}\textcolor{gray}{\}}</p>
    </\textcolor{gray}{\{}\textcolor{blue}{\{\{Input[1]\}\}}\textcolor{gray}{, }\textcolor{blue}{body}\textcolor{gray}{\}}>
</html>
\textcolor{gray}{\}}
\end{data}
\caption{Generic Theory}
\label{fig:generic_theory}
\end{figure}

Note that in this case, the HTML body tags are considered to have been potentially derived from the input.
Of course, with another input-output example, this program space could be collapsed to resolve this as discussed in Section \ref{sec:theory_intersection}.
Nevertheless, it is useful for demonstrating how much extra work must be needlessly done in the absence of domain-specific knowledge about the HTML structure of this program's output.

\section{Conversation Intersection}

The prior structure of the generic theorist (Figure \ref{fig:generic_theorist}) does not scale very well with message sizes and the number of core theorists (e.g. $FunctionTheorist$s).
Consider for instance a message of 100,000 bytes, with hundreds of potential $FunctionTheorist$s and $CaptureTheorist$s defined to allow for all sorts of context derivations.
This would likely be the case in applying IRE to real world web applications, where HTML pages get quite large, and context derivations quite complex.
Under such circumstances, applying this sort of generic theorist would become infeasible.

Fortunately, IRE can do much better by analyzing multiple conversations at once, prior to producing their theories and intersecting them.
With this early access, IRE can perform a sort of conversation intersection before it begins its traditional analysis.
In doing so, IRE analyzes the supplied conversations to determine their structure.
In particular, it looks for portions of messages which remain constant throughout all of the conversations.
IRE uses the insight that if portions of a message don't change across any conversation examples, then it must be the case that a consistent program trace keeps it constant.
Boldly, IRE assumes that it must be constant.

Effectively, IRE automatically generates a grammar, where constant sections are parsed by constant theorists, and nonconstant sections are parsed by all supplied core theorists.
This can dramatically decrease the program search space.
However, IRE must first determine which sections of messages are constant, and which aren't.

This is done by taking all corresponding messages across all conversations, and finding the longest common substring across them.
The longest common substring is marked as being constant in the grammar, and this is recursively applied to all the prefixes and all the suffixes.
This continues on until there no longer is a common substring.
This ultimately results in marking the maximum amount of constant data across all conversations.

Observe what happens (Figure \ref{fig:conversation_intersection}) when the output from input \mintinline{python}{name='Paul'} and \mintinline{python}{msg='Hello'} and output from input \mintinline{python}{name='Pablo'} and \mintinline{python}{msg='Hola'} run against the prior simple program (Figure \ref{fig:simple_html_echo_program}) are intersected.

\begin{figure}[tb]
\begin{data}
\let =\enskip
<html>
    <head>
        <title>Echo</title>
    </head>
    <body>
        <p>Hello Pa\summaryii{u}{b} l\summaryii{}{o}!</p>
        <p>"H\summaryii{ell}{} o\summaryii{}{la} " is SG\summaryii{V}{9} s\summaryii{bG8}{VQ=} =</p>
    </body>
</html>
\end{data}
\caption{Conversation Intersection}
\label{fig:conversation_intersection}
\end{figure}

This shows an interesting result common to intersecting only a few examples: too much of the conversation is marked constant.
This can potentially cause problems with inputs being fragmented, though this is not necessarily a major issue.
Fortunately, though, with more examples this problem is resolved.
Figure \ref{fig:another_conversation_intersection} shows the result of introducing another output due to input \mintinline{python}{name='Joe'} and \mintinline{python}{msg='ABC'}.

\begin{figure}[tb]
\begin{data}
\let =\enskip
<html>
    <head>
        <title>Echo</title>
    </head>
    <body>
        <p>Hello \summaryiii{Paul}{Pablo}{Joe}!</p>
        <p>"\summaryiii{Hello}{Hola}{ABC}" is \summaryiii{SGVsbG8=}{SG9sYQ==}{QUJD}</p>
    </body>
</html>
\end{data}
\caption{Another Conversation Intersection}
\label{fig:another_conversation_intersection}
\end{figure}

This discovered structure (Figure \ref{fig:another_conversation_intersection}), in turn, is used by IRE for automatically generating a grammar.
This dramatically reduces the possible program space, enabling IRE to be much more efficient.

\chapter{Related Work}

Early work in this field has formalized inductive learning as a search problem \cite{mitchell1982generalization}.
Past work has shown just how useful version space algebras can be in representing a program space \cite{hirsh1991theoretical, lau2000version, lau2003programming}.
Several search techniques including enumerative, stochastic, and constraint-based algorithms have been developed in order to search the program space in a number of domains \cite{udupa2013transit, schkufza2013stochastic, solar2008program, feser2015synthesizing}.

Microsoft has made major contributions in this area.
Much of the research discussed here builds upon their work in Inductive Program Synthesis.
Their \textit{PROSE SDK} enables program synthesis development in a way akin to what this research explores \cite{gulwani2016programming}.
Their \textit{FlashFill} program demonstrates some of the potential applications that this research has for millions of users \cite{gulwani2011automating, gulwani2015inductive}.

The specific problem of protocol reverse engineering is a very closely related area that has been researched in order to quickly understand custom botnet protocols in \textit{Prospex} and \textit{Dispatcher} \cite{comparetti2009prospex, caballero2009dispatcher}.

Because this topic can be applied to so many areas, research on it is scattered in several fields \cite{kitzelmann2009inductive}.
For this reason, IRE aspires to be a centralized location and interface for implementing program synthesis and automatic black box analysis techniques.

%%%%%%%%%%%%%%%%%%%%%%%%%%%%%%%%%%%%%%%
% Back matter
%%%%%%%%%%%%%%%%%%%%%%%%%%%%%%%%%%%%%%%
\SingleSpacing                          % Back matter should be single spaced

\edef\defaulttolerance{\the\tolerance}
\tolerance 500                          % Increase tolerance to prevent material extending into margins
\hbadness 500

\iftoggle{useendnotes}{%                % If you're using endnotes, output them here
  \setsecnumdepth{none}                 % No section numbering in end notes
  \printpagenotes
  \setsecnumdepth{all}%                 % Turn section numbering back on after printing
}{}

\chapter*{\bibheading}                  % In the running text, use a chapter-level heading
                                        % for the bibliography section
\phantomsection
\addcontentsline{toc}{chapter}{%        % In the TOC, add a custom chapter-level heading
  \hspace{-\cftchapterindent}%          % that will be flush against the left margin
  \bibheading%
}
%\phantomsection
%\addtocontents{toc}%                    % Add this 'mark' to TOC so subsequent pages use
%  {\protect\markboth{\bibheading}{Page}}%   the bibliography heading (unlikely since
%                                        %   the appendices follow quickly)
\iftoggle{usebiblatex}{%                % Output the bibliography
  \printbibliography[heading=none]      % Using a 'biblatex' package; do not let
                                        %   'biblatex' output a heading
}{%
  \renewcommand\bibsection{}            % Do not let 'natbib' output a heading
  \bibliographystyle{\natbibstyle}      % Using 'natbib' to print bibliography
  \bibliography{\bibfilename}
}

% \appendix                             % Indicate start of appendices
                                        % Appendices are considered 'mainmatter' in this
                                        %   documentclass
\tolerance \defaulttolerance            % Set tolerance back to default
\hbadness \defaulttolerance

\addtocontents{toc}{\protect%           % Only include appendix title in table of contents
  \setcounter{tocdepth}{0}}%            %   and omit sub-headings
\renewcommand*{\chapnamefont}%          % Reset font for 'Appendix' in chapter titles
    {\normalfont\MakeTextUppercase}
\makeatletter                           % Clear page after printing appendix title
  \renewcommand{\memendofchapterhook}%
  {%
    \clearpage
    \m@mindentafterchapter
    \@afterheading
  }
\makeatother

\phantomsection                         % Need '\phantomsection' to place hyperref
                                        %   bookmark more accurately
%\addcontentsline{toc}{part}{Appendix}  %~Add "Appendix" to TOC here; comment out this
                                        %   line if you're not including appendices

%\phantomsection                        %!This is the one part of the template that I
%\addtocontents{toc}%                   %   could not get to work properly. After you
%  {\protect\markboth{APPENDIX}{Page}}  %   start listing appendices in the TOC,
                                        %   subsequent TOC pages should use "APPENDIX in
                                        %   the header instead of "CHAPTER"; however,
                                        %   this code will make "APPENDIX" appear on the
                                        %   the same page that the *first* appendix
                                        %   appears on. This problem won't affect most
                                        %   people, but if it affects you, uncomment
                                        %   these lines and move them below where
                                        %   the appendices are listed. Keep moving these
                                        %   lines down and checking the output until
                                        %   the TOC headers appear correctly

%\include{appendix1}                     %~Insert your appendices here; I recommend to use
%\include{appendix2}                     %   \include rather than \input for appendices.
%\include{appendix3}% etc.               %   All heading commands are the same as above,
                                         %   e.g., \chapter, \section, etc.

\backmatter                             % Start back matter according to documentclass
\makeatletter                           % Do not clear page after printing title for
  \renewcommand{\memendofchapterhook}%  %   biographical sketch
  {%
    \m@mindentafterchapter
    \@afterheading
  }
\makeatother
%\chapter{Biographical Sketch}           %~Biographical Sketch is optional
%\input{biography}                       %<Enter the name of the .tex file containing your
                                        %   biography or omit this line and type in
                                        %   your biography here (1 paragraph)

\end{document}
