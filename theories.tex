\chapter{Theorists and Theories}

IRE works around two central primitives: theorists and theories.
Theorists use input-output examples to produce theories.
They act as a sort of domain-specific language.
Theories use input to produce output.
They model the synthesized program.

\image{theorist.png}{Theorist}{0.7}

\image{theory.png}{Theory}{0.7}

\section{Grammar Theorists}

IRE fundamentally solves a parsing problem.
It must parse input-output examples to produce a program, not entirely dissimilar to a programming language parser which parses source code to produce a program.
For this reason, it is useful to think in terms of context-free grammars.
While in the programming language realm context-free grammars strive to be unambiguous, IRE leverages ambiguity to produce the program space.
Consider, for instance, the following HTML grammar theorist (Figure \ref{fig:html_grammar}) written using the IRE framework.

\begin{figure}[H]
\begin{pythoncode}
import ire

class HTML(ire.GrammarTheorist, entry='html'):
    html = ire.RepeatTheorist(html_element) ~$\label{html_theorist}$~
    html_element = (open_tag & close_tag & self_tag) | data ~$\label{html_element_theorist}$~

    open_tag = '<' + tag_content + '>'
    close_tag = '</' + tag_name + '>'
    self_tag = '<' + tag_content + '/>'

    data = '.|\n' & ire.FunctionTheorist(None) & ire.FunctionTheorist(b64) ~$\label{data_theorist}$~

    tag_content = (tag_name + whitespace + attributes) | tag_name
    tag_name = '[a-zA-Z]+'

    attributes = (attribute + whitespace + attributes) | attribute
    attribute = '[a-zA-Z]+' & ('[a-zA-Z]+="' + attribute_value + '"')
    attribute_value = '[a-zA-Z0-9]'

    whitespace = '\s+'
\end{pythoncode}
\caption{HTML Grammar Theorist}
\label{fig:html_grammar}
\end{figure}

Here, a $GrammarTheorist$ is defined, with $html$ being the entry point, or start symbol in context-free grammar terminology.
Variables that appear on the left side of an $=$ indicate a nonterminal symbol, while all other expressions are terminal symbols.
Each of these assignments forms the basis for a production rule.

Strings are implicitly $RegexTheorist$s, which as the name suggests, perform regex matching.
Addition expressions (using the $+$ operator) are implicitly $ConcatTheorist$s, which serve to concatenate theorists, and their produced theories, together.

Rather than defining several production rules for the same nonterminal, operators $\&$ and $|$ both respectively serve to implicitly create $AndTheorist$s and $OrTheorist$s.
All theorists within an $AndTheorist$ will attempt to produce theories.
After some theorist within an $OrTheorist$ has produced some number of theories, any remaining theorists will not be given a chance to produce theories.
This evaluation takes place from left to right and allows for a conditionally restricted search space, and consequently more efficient parsing.
On line \ref{html_element_theorist}, the $OrTheorist$ indicates that an $html\_element$ can be an $open\_tag$, $close\_tag$ and $self\_tag$, or if that is not the case then it must be $data$.

On line \ref{html_theorist}, the $RepeatTheorist$ indicates that parsing should repeatedly consume (one or more times) $html\_element$s until it no longer can.
Implementation wise, this is more efficient than how looping is traditionally performed in context-free grammars by having a self-referential production rule.

$FunctionTheorist$s wrap functions to make them behave as theorists.
Line \ref{data_theorist} showcases two of these theorists.
$FunctionTheorist(None)$ indicates the identity function ($f(x)=x$).
$FunctionTheorist(b64)$ on the other hand assumes the existence of a $b64$ function (defined elsewhere), which base64 encodes its input.


\section{Theorists to Theories}

Consider the following simple HTML echo program (Figure \ref{fig:simple_html_echo_program}) in order to understand how IRE is able to utilize the prior HTML grammar theorist (Figure \ref{fig:html_grammar}) to reason about this program and produce theories.

\begin{figure}[H]
\begin{pythoncode}
def echo(name, msg):
    return \
f"""
<html>
    <head>
        <title>Echo</title>
    </head>
    <body>
        <p>Hello {name}!</p>
        <p>"{msg}" is {b64(msg)}</p>
    </body>
</html>
"""
\end{pythoncode}
\caption{Simple HTML Echo Program}
\label{fig:simple_html_echo_program}
\end{figure}

This simple program (Figure \ref{fig:simple_html_echo_program}) transforms the inputs into an HTML output.
Inputs \mintinline{python}{name='Paul'} and \mintinline{python}{msg='Hello World'} result in the following output (Figure \ref{fig:output_of_simple_html_echo_program}).

\begin{figure}[H]
\begin{data}
\let =\enskip
<html>
    <head>
        <title>Echo</title>
    </head>
    <body>
        <p>Hello Paul!</p>
        <p>"Hello World" is SGVsbG8gV29ybGQ=</p>
    </body>
</html>
\end{data}
\caption{Output of Simple HTML Echo Program}
\label{fig:output_of_simple_html_echo_program}
\end{figure}

Running this input-output example (Figure \ref{fig:output_of_simple_html_echo_program}) through the prior HTML grammar theorist (Figure \ref{fig:html_grammar}) results in the following theory (Figure \ref{fig:simple_html_echo_theory}), displayed as a simple program summary using the IRE framework.

\begin{figure}[H]
\begin{data}
\let =\enskip
\textcolor{gray}{\{}
<html>
    <head>
        <title>Echo</title>
    </head>
    <body>
        <p>Hello \textcolor{gray}{\{}\textcolor{blue}{\{\{Input[0]\}\}}\textcolor{gray}{, }\textcolor{blue}{Paul}\textcolor{gray}{\}}!</p>
        <p>"\textcolor{gray}{\{}\textcolor{blue}{\{\{Input[1]\}\}}\textcolor{gray}{, }\textcolor{blue}{Hello World}\textcolor{gray}{\}}" is \textcolor{gray}{\{}\textcolor{blue}{\{\{b64(Input[1])\}\}}\textcolor{gray}{, }\textcolor{blue}{SGVsbG8gV29ybGQ=}\textcolor{gray}{\}}</p>
    </body>
</html>
\textcolor{gray}{\}}
\end{data}
\caption{Simple HTML Echo Theory}
\label{fig:simple_html_echo_theory}
\end{figure}

In this simple program summary (Figure \ref{fig:simple_html_echo_theory}), the gray curly braces and commas represent $UnionTheory$s, and the blue text represents individual theories within those $UnionTheory$s.
$UnionTheory$s are a way of succinctly representing the program space, allowing common theories among candidate program traces to not be repeated, effectively forming a directed acyclic graph.
Double curly braces indicate theories that depend on the input (e.g. $FunctionTheory$s).

This shows an interesting result common to running a theorist against only one input-output example: the complete original constant output by itself appears as a possible program trace within the program space.
These constant theories can be ruled out only with more input-output examples.

\section{Theory Intersection}
\label{sec:theory_intersection}

In order to collapse the program space, theories resulting from different input-output examples must be intersected together.

\image{intersection.png}{Theory Intersection}{0.7}

Inputs \mintinline{python}{name='Pablo'} and \mintinline{python}{msg='Hola Mundo'} result in the following output (Figure \ref{fig:another_output_of_simple_html_echo_program}).

\begin{figure}[H]
\begin{data}
\let =\enskip
<html>
    <head>
        <title>Echo</title>
    </head>
    <body>
        <p>Hello Pablo!</p>
        <p>"Hola Mundo" is SG9sYSBNdW5kbw==</p>
    </body>
</html>
\end{data}
\caption{Another Output of Simple HTML Echo Program}
\label{fig:another_output_of_simple_html_echo_program}
\end{figure}

Running this input-output example (Figure \ref{fig:another_output_of_simple_html_echo_program}) through the prior HTML grammar theorist (Figure \ref{fig:html_grammar}) results in the following theory (Figure \ref{fig:another_simple_html_echo_theory}), displayed as a simple program summary using the IRE framework.

\begin{figure}[H]
\begin{data}
\let =\enskip
\textcolor{gray}{\{}
<html>
    <head>
        <title>Echo</title>
    </head>
    <body>
        <p>Hello \textcolor{gray}{\{}\textcolor{blue}{\{\{Input[0]\}\}}\textcolor{gray}{, }\textcolor{blue}{Pablo}\textcolor{gray}{\}}!</p>
        <p>"\textcolor{gray}{\{}\textcolor{blue}{\{\{Input[1]\}\}}\textcolor{gray}{, }\textcolor{blue}{Hola Mundo}\textcolor{gray}{\}}" is \textcolor{gray}{\{}\textcolor{blue}{\{\{b64(Input[1])\}\}}\textcolor{gray}{, }\textcolor{blue}{SG9sYSBNdW5kbw==}\textcolor{gray}{\}}</p>
    </body>
</html>
\textcolor{gray}{\}}
\end{data}
\caption{Another Simple HTML Echo Theory}
\label{fig:another_simple_html_echo_theory}
\end{figure}

Intersecting the first theory (Figure \ref{fig:simple_html_echo_theory}) with the second theory (Figure \ref{fig:another_simple_html_echo_theory}), results in just one program trace in the program space: our original program (Figure \ref{fig:simple_html_echo_program}).

\begin{figure}[H]
\begin{data}
\let =\enskip
\textcolor{gray}{\{}
<html>
    <head>
        <title>Echo</title>
    </head>
    <body>
        <p>Hello \textcolor{gray}{\{}\textcolor{blue}{\{\{Input[0]\}\}}\textcolor{gray}{\}}!</p>
        <p>"\textcolor{gray}{\{}\textcolor{blue}{\{\{Input[1]\}\}}\textcolor{gray}{\}}" is \textcolor{gray}{\{}\textcolor{blue}{\{\{b64(Input[1])\}\}}\textcolor{gray}{\}}</p>
    </body>
</html>
\textcolor{gray}{\}}
\end{data}
\caption{Simple HTML Echo Theory Intersected}
\label{fig:simple_html_echo_theory_intersected}
\end{figure}
