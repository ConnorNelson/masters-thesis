\chapter{Introduction}

Reverse engineering is a methodology for precisely analyzing the internal workings and substructure of a process or system in order to better understand how it works.
In practice, it is done in the absence of high-level specifications and can be thought of as working backward through the standard engineering process---design towards implementation---and instead, implementation towards design.
In theory, reverse engineering is relatively straightforward: simply observe how the internals are operating and how the subcomponents are connected.
Of course, it is more nuanced than this; but even so, a complete working system that can be observed is, by its very nature, perfectly descriptive of what it does and how it works.
This is an underlying requirement of standard reverse engineering.
In hardware, a physical object is disassembled and examined.
In software, its source code is read through, or in its absence, binary disassembled and machine code analyzed.

Consider, however, the task of reverse engineering without complete access to observing the system.
This is the case in distributed systems, where some agent has only partial access to the overall system.
Take for instance web applications, where a client makes a request to a server and is only made aware of its response.
From the client's perspective, the server is merely a black-box---an oracle---that takes some input and returns some output.
What happens in between is left unknown to the client.
In such cases, reverse engineering becomes inherently uncertain.

Further consider the problem of systems in which interactions take place between persons and computers; for instance, a human interacting with some computer program in a repetitive way.
This constitutes a distributed system, where part of the program takes place in the computer, but also part of it takes place in the person's intentions towards interacting with the computer.
In such cases, the program occurring in the person's intentions---in the person's mind---is but a black-box to the computer.
It is in this way that reverse engineering may be applied not only towards analyzing a computer, but also a person.
Reverse engineering may be useful here in order to profile or improve upon the user experience of the person.

Although reverse engineering in these situations becomes an inherently uncertain process, this does not stop human-efforts in reverse engineering.
In practice, humans build up an entire model of the black-box system under analysis.
They make assumptions based on past systems, attempt to rule out and confirm these assumptions, and use intuition as a means of guiding this process.
This inductive reasoning forms the basis for inductive reverse engineering: IRE.

IRE serves to solve this problem of reverse engineering in a black-box environment.
IRE is an easy to use, open source, Python 3 framework, that enables users to transform input-output examples into executable programs consistent with those examples.
This effectively allows for programming by example.
Users can easily introduce domain-specific knowledge about the problem they are working on to further enhance this process.
