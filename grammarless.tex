\chapter{Grammarless Synthesis}

While it is useful to be able to provide domain-specific knowledge to an analysis, doing so is not always possible.
Providing a grammar to IRE allows it to be more efficient in its search, by reducing the possible program space.
Consider, for instance, the grammar shown in Figure \ref{fig:generic_theorist} which is highly generic.

\begin{figure}[tb]
\begin{pythoncode}
class Generic(ire.GrammarTheorist, entry='entry'):
    entry = ire.RepeatTheorist(data)
    data = '.|\n' & ire.FunctionTheorist(None) & ire.FunctionTheorist(b64)
\end{pythoncode}
\caption{Generic Theorist}
\label{fig:generic_theorist}
\end{figure}

This grammar (Figure \ref{fig:generic_theorist}) repeatedly matches against a single byte and derivations of the input.
Repeatedly matching against a single byte is what effectively drives the parsing forward.

Inputs \mintinline{python}{name='Paul'} and \mintinline{python}{msg='body'} run against the prior simple program (Figure \ref{fig:simple_html_echo_program}) and through the generic theorist (Figure \ref{fig:generic_theorist}) result in the theory shown in Figure \ref{fig:generic_theory}, displayed as a simple program summary using the IRE framework.

\begin{figure}[tb]
\begin{data}
\let =\enskip
\textcolor{gray}{\{}
<html>
    <head>
        <title>Echo</title>
    </head>
    <\textcolor{gray}{\{}\textcolor{blue}{\{\{Input[1]\}\}}\textcolor{gray}{, }\textcolor{blue}{body}\textcolor{gray}{\}}>
        <p>Hello \textcolor{gray}{\{}\textcolor{blue}{\{\{Input[0]\}\}}\textcolor{gray}{, }\textcolor{blue}{Paul}\textcolor{gray}{\}}!</p>
        <p>"\textcolor{gray}{\{}\textcolor{blue}{\{\{Input[1]\}\}}\textcolor{gray}{, }\textcolor{blue}{body}\textcolor{gray}{\}}" is \textcolor{gray}{\{}\textcolor{blue}{\{\{b64(Input[1])\}\}}\textcolor{gray}{, }\textcolor{blue}{Ym9keQ==}\textcolor{gray}{\}}</p>
    </\textcolor{gray}{\{}\textcolor{blue}{\{\{Input[1]\}\}}\textcolor{gray}{, }\textcolor{blue}{body}\textcolor{gray}{\}}>
</html>
\textcolor{gray}{\}}
\end{data}
\caption{Generic Theory}
\label{fig:generic_theory}
\end{figure}

Note that in this case, the HTML body tags are considered to have been potentially derived from the input.
Of course, with another input-output example, this program space could be collapsed to resolve this as discussed in Section \ref{sec:theory_intersection}.
Nevertheless, it is useful for demonstrating how much extra work must be needlessly done in the absence of domain-specific knowledge about the HTML structure of this program's output.

\section{Conversation Intersection}

The prior structure of the generic theorist (Figure \ref{fig:generic_theorist}) does not scale very well with message sizes and the number of core theorists (e.g. $FunctionTheorist$s).
Consider for instance a message of 100,000 bytes, with hundreds of potential $FunctionTheorist$s and $CaptureTheorist$s defined to allow for all sorts of context derivations.
This would likely be the case in applying IRE to real world web applications, where HTML pages get quite large, and context derivations quite complex.
Under such circumstances, applying this sort of generic theorist would become infeasible.

Fortunately, IRE can do much better by analyzing multiple conversations at once, prior to producing their theories and intersecting them.
With this early access, IRE can perform a sort of conversation intersection before it begins its traditional analysis.
In doing so, IRE analyzes the supplied conversations to determine their structure.
In particular, it looks for portions of messages which remain constant throughout all of the conversations.
IRE uses the insight that if portions of a message don't change across any conversation examples, then it must be the case that a consistent program trace keeps it constant.
Boldly, IRE assumes that it must be constant.

Effectively, IRE automatically generates a grammar, where constant sections are parsed by constant theorists, and nonconstant sections are parsed by all supplied core theorists.
This can dramatically decrease the program search space.
However, IRE must first determine which sections of messages are constant, and which aren't.

This is done by taking all corresponding messages across all conversations, and finding the longest common substring across them.
The longest common substring is marked as being constant in the grammar, and this is recursively applied to all the prefixes and all the suffixes.
This continues on until there no longer is a common substring.
This ultimately results in marking the maximum amount of constant data across all conversations.

Observe what happens (Figure \ref{fig:conversation_intersection}) when the output from input \mintinline{python}{name='Paul'} and \mintinline{python}{msg='Hello'} and output from input \mintinline{python}{name='Pablo'} and \mintinline{python}{msg='Hola'} run against the prior simple program (Figure \ref{fig:simple_html_echo_program}) are intersected.

\begin{figure}[tb]
\begin{data}
\let =\enskip
<html>
    <head>
        <title>Echo</title>
    </head>
    <body>
        <p>Hello Pa\summaryii{u}{b} l\summaryii{}{o}!</p>
        <p>"H\summaryii{ell}{} o\summaryii{}{la} " is SG\summaryii{V}{9} s\summaryii{bG8}{VQ=} =</p>
    </body>
</html>
\end{data}
\caption{Conversation Intersection}
\label{fig:conversation_intersection}
\end{figure}

This shows an interesting result common to intersecting only a few examples: too much of the conversation is marked constant.
This can potentially cause problems with inputs being fragmented, though this is not necessarily a major issue.
Fortunately, though, with more examples this problem is resolved.
Figure \ref{fig:another_conversation_intersection} shows the result of introducing another output due to input \mintinline{python}{name='Joe'} and \mintinline{python}{msg='ABC'}.

\begin{figure}[tb]
\begin{data}
\let =\enskip
<html>
    <head>
        <title>Echo</title>
    </head>
    <body>
        <p>Hello \summaryiii{Paul}{Pablo}{Joe}!</p>
        <p>"\summaryiii{Hello}{Hola}{ABC}" is \summaryiii{SGVsbG8=}{SG9sYQ==}{QUJD}</p>
    </body>
</html>
\end{data}
\caption{Another Conversation Intersection}
\label{fig:another_conversation_intersection}
\end{figure}

This discovered structure (Figure \ref{fig:another_conversation_intersection}), in turn, is used by IRE for automatically generating a grammar.
This dramatically reduces the possible program space, enabling IRE to be much more efficient.
