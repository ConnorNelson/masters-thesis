\chapter{Conversations}

Often times, distributed programs follow a conversation-like behavior, potentially following some underlying protocol.
In such cases, program synthesis cannot merely take place over only simple input-output examples.
Instead, input-output examples must be generalized to conversation examples.
Here, discerning between input and output doesn't necessarily make sense, as it depends on an agent's perspective.
Instead, a conversation takes place over a series of messages, where a message has some source agent and destination agent.
Further, these agents have context--a generalization of input--representing what an agent individually brings as their input to a particular conversation.
This allows for some notion of state.
Nevertheless, in a deterministic program, identical contexts will produce identical conversations.

\image{conversation.png}{Conversation}{0.5}

\section{Capturing Theorists and Theories}

An agent is not aware of other agents' contexts; only its own context and any received messages.
This is a fundamental problem in reverse engineering distributed systems.
Therefore, IRE must use an agent's received messages in order to recover the sender's context.
It does so using $CaptureTheorist$s and $CaptureTheory$s.

$CaptureTheorist$s wrap other theorists and effectively leverage their ability to parse.
Those $CaptureTheorist$s go on to create $CaptureTheory$s, which also wrap those same theorists.
During synthesis, the underlying theorists attempt to produce theories, and anything consumed during this is captured.
This captured data may then be used in later theorists, much like a traditional input.
Those produced theories do not actually become a part of the program space, but instead the $CaptureTheory$s storing the underlying theorist does.
This allows this parsing done during the synthesis to be performed again during execution of the theory.

\image{capture.png}{Capture Theorist and Capture Theory}{0.7}

Consider the HTTP theorists shown in Figure \ref{fig:http_theorist}, written using the IRE framework, in order to understand how these $CaptureTheorist$s and $CaptureTheory$s may be applied.

\begin{figure}[tb]
\begin{pythoncode}
class HTTPRequest(ire.OutputMessageTheorist, entry='request'):
    request = request_prolog + headers + '\n' + request_contents
    request_prolog = '(GET|POST) ' + request_url + ' HTTP/1.1\n'
    request_url = ire.RepeatTheorist(request_url_data)
    request_url_data = '[^ ]' & inputs

    headers = ire.RepeatTheorist(header)
    header = '[a-zA-Z-]+: ' + ire.RepeatTheorist(header_data) + '\n'
    header_data = '.' & inputs

    request_contents = ire.RepeatTheorist(request_contents_data) | ''
    request_contents_data = '.|\n' & inputs

    inputs = ire.FunctionTheorist(None) & ire.FunctionTheorist(b64)


class HTTPResponse(ire.InputMessageTheorist, entry='response'):
    response = response_prolog + headers + '\n' + response_contents
    response_prolog = 'HTTP/1.1 [0-9]+ [a-zA-Z ]+\n'

    headers = ire.RepeatTheorist(cookie_header | header)
    cookie_header = 'Set-Cookie: ' + cookie + '\n'
    header = '[a-zA-Z-]+: ' + ire.RepeatTheorist(header_data) + '\n'
    header_data = '.' & inputs

    response_contents = HTML() | ''

    cookie = ire.CaptureTheorist(cookie_data, 'cookie') ~$\label{cookie_theorist}$~
    cookie_data = '[a-zA-Z]+=[a-zA-Z0-9]+'

    inputs = ire.FunctionTheorist(None) & ire.FunctionTheorist(b64)


http = HTTPRequest() & HTTPResponse()
\end{pythoncode}
\caption{HTTP Theorist}
\label{fig:http_theorist}
\end{figure}

In the case of web applications, it is common for an HTTP server to negotiate some token, commonly known as a cookie, for keeping track of state with its clients \cite{barth2011rfc}.
Cookies are a way of enabling statefulness in the otherwise stateless HTTP protocol.
An HTTP client will include the cookies associated with a particular server with all web requests made to that server.
By introducing a $CaptureTheorist$ on line \ref{cookie_theorist}, this behavior is effectively conveyed and enables IRE to capture the cookie.

Now consider the HTTP conversation (Figure \ref{fig:simple_http_conversation}) that results from the client's context of \mintinline{python}{username='Paul'}, \mintinline{python}{password='p455w0rd'}, and \mintinline{python}{msg='Hello_World'}; and unknown to the client, server's context of \mintinline{python}{cookie='sessionid=12345'}.

\begin{figure}[tb]
\begin{data}
\let =\enskip
\begin{data}
POST /login HTTP/1.1
Host: example.com
Content-Type: application/json
\medskip
\{"username": "Paul", "password": "p455w0rd"\}
\end{data}

\begin{data}
HTTP/1.1 200 OK
Set-Cookie: sessionid=12345
Connection: close
\medskip
\end{data}

\begin{data}
GET /echo?msg=Hello\_World HTTP/1.1
Host: example.com
Cookie: sessionid=12345
\medskip
\end{data}

\begin{data}
HTTP/1.1 200 OK
Content-Type: text/html; charset=UTF-8
Connection: close
\medskip
<html>
    <head>
        <title>Echo base64</title>
    </head>
    <body>
        <p>Hello Paul!</p>
        <p>"Hello\_World" is SGVsbG9fV29ybGQ=</p>
    </body>
</html>
\end{data}
\end{data}
\caption{Simple HTTP Conversation}
\label{fig:simple_http_conversation}
\end{figure}

Running this conversation example (Figure \ref{fig:simple_http_conversation}) through the prior HTTP theorist (Figure \ref{fig:http_theorist}) results in the theory shown in Figure \ref{fig:simple_http_theory}, displayed as a simple program summary using the IRE framework.

\begin{figure}[tb]
\begin{data}
\let =\enskip
\begin{data}
\textcolor{gray}{\{}
POST /login HTTP/1.1
Host: example.com
Content-Type: application/json
\medskip
\{"username": "\textcolor{gray}{\{}\textcolor{blue}{\{\{Input[username]\}\}}\textcolor{gray}{, }\textcolor{blue}{Paul}\textcolor{gray}{\}}", "password": "\textcolor{gray}{\{}\textcolor{blue}{\{\{Input[password]\}\}}\textcolor{gray}{, }\textcolor{blue}{p455w0rd}\textcolor{gray}{\}}"\}
\textcolor{gray}{\}}
\end{data}

\begin{data}
\textcolor{gray}{\{}
HTTP/1.1 200 OK
Set-Cookie: \textcolor{gray}{\{}\textcolor{blue}{\{\{Capture[cookie]\}\}}\textcolor{gray}{\}}
Connection: close
\medskip
\textcolor{gray}{\}}
\end{data}

\begin{data}
\textcolor{gray}{\{}
GET /echo?msg=\textcolor{gray}{\{}\textcolor{blue}{\{\{Input[msg]\}\}}\textcolor{gray}{, }\textcolor{blue}{Hello\_World}\textcolor{gray}{\}} HTTP/1.1
Host: example.com
Cookie: \textcolor{gray}{\{}\textcolor{blue}{\{\{Input[cookie]\}\}}\textcolor{gray}{, }\textcolor{blue}{sessionid=12345}\textcolor{gray}{\}}
\medskip
\textcolor{gray}{\}}
\end{data}

\begin{data}
\textcolor{gray}{\{}
HTTP/1.1 200 OK
Content-Type: text/html; charset=UTF-8
Connection: close
\medskip
<html>
    <head>
        <title>Echo base64</title>
    </head>
    <body>
        <p>Hello \textcolor{gray}{\{}\textcolor{blue}{\{\{Input[username]\}\}}\textcolor{gray}{, }\textcolor{blue}{Paul}\textcolor{gray}{\}}!</p>
        <p>"\textcolor{gray}{\{}\textcolor{blue}{\{\{Input[msg]\}\}}\textcolor{gray}{, }\textcolor{blue}{Hello\_World}\textcolor{gray}{\}}" is \textcolor{gray}{\{}\textcolor{blue}{\{\{b64(Input[username])\}\}}\textcolor{gray}{, }\textcolor{blue}{SGVsbG9fV29ybGQ=}\textcolor{gray}{\}}</p>
    </body>
</html>
\textcolor{gray}{\}}
\end{data}
\end{data}
\caption{Simple HTTP Theory}
\label{fig:simple_http_theory}
\end{figure}

With another example conversation, this program space could be collapsed as discussed in Section \ref{sec:theory_intersection}.
