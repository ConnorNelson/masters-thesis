\chapter{Related Work}

Early work in this field has formalized inductive learning as a search problem \cite{mitchell1982generalization}.
Past work has shown just how useful version space algebras can be in representing a program space \cite{hirsh1991theoretical, lau2000version, lau2003programming}.
Several search techniques including enumerative, stochastic, and constraint-based algorithms have been developed in order to search the program space in a number of domains \cite{udupa2013transit, schkufza2013stochastic, solar2008program, feser2015synthesizing}.

Microsoft has made major contributions in this area.
Much of the research discussed here builds upon their work in Inductive Program Synthesis.
Their \textit{PROSE SDK} enables program synthesis development in a way akin to what this research explores \cite{gulwani2016programming}.
Their \textit{FlashFill} program demonstrates some of the potential applications that this research has for millions of users \cite{gulwani2011automating, gulwani2015inductive}.

The specific problem of protocol reverse engineering is a very closely related area that has been researched in order to quickly understand custom botnet protocols in \textit{Prospex} and \textit{Dispatcher} \cite{comparetti2009prospex, caballero2009dispatcher}.

Because this topic can be applied to so many areas, research on it is scattered in several fields \cite{kitzelmann2009inductive}.
For this reason, IRE aspires to be a centralized location and interface for implementing program synthesis and automatic black box analysis techniques.
